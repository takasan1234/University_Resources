\documentclass[a4j]{jsarticle}
\usepackage[dvipdfmx]{graphicx}
\usepackage{amsmath}
\usepackage{url}

\title{相関ルールマイニングの概要}
\author{大阪大学 工学部 電子情報学科 3年\\情報システム工学コース\\08D23091\,辻\,孝弥}
\date{\today}

\begin{document}
\maketitle

\section{はじめに}

相関ルールマイニング(Association Rule Mining)は,大量の取引データ(トランザクション)から「あるアイテムの組み合わせが同時に出現しやすい」というパターンを発見する手法である.小売業におけるバスケット分析に端を発し,マーケティング施策やレコメンデーションの基盤として広く用いられている.

\section{基本概念}

\subsection{トランザクションとアイテム集合}

\begin{description}
  \item[トランザクション:] 顧客の購買履歴など,アイテムの集合を表す(例:\{ビール, おむつ, パン\}).
  \item[アイテム集合(Itemset):] 複数のアイテムをまとめた集合.サイズ$k$のアイテム集合を$k$-アイテム集合と呼ぶ.
\end{description}

\subsection{指標定義}

\begin{description}
  \item[支持度 (Support):] 項目集合$X$が出現する頻度を表す.
        \[
          P(X)=\frac{|\{T \mid X \subseteq T\}|}{\text{全トランザクション数}}
        \]
  \item[信頼度 (Confidence):] $X$を含むトランザクションのうち$Y$も含む割合.
        \[
          P(Y|X)=\frac{P(X \cup Y)}{P(X)}
        \]
  \item[リフト (Lift):] 期待値以上に同時出現する強さ.1以上で正の相関.
        \[
          \frac{P(Y|X)}{P(Y)}
        \]
\end{description}

\section{アルゴリズム概要}

\subsection{Aprioriアルゴリズム}

\begin{enumerate}
  \item L1: すべての1-アイテム集合の支持度を計算し,最小支持度(minsup)以上の集合を抽出.
  \item 候補生成 (Ck): 頻出(k--1)-アイテム集合同士を結合し,k-アイテム集合候補を作成.
  \item プルーニング: 候補の部分集合がすべて頻出でないものは除去.
  \item 支持度計算: トランザクションを再スキャンし,Ckの支持度を数えてLkを得る.
  \item kを増やし,Lkが空になるまで繰り返す.
\end{enumerate}

\subsection{ルール生成}

各頻出アイテム集合$\ell$をその非空部分集合$s$と$\ell-s$に分割し,ルール候補 $s \Rightarrow (\ell-s)$ を作成.
信頼度を計算し,最小信頼度(minconf)以上のルールを採択.

\section{応用例と利点・課題}

\subsection{応用例}
\begin{itemize}
  \item 小売業のバスケット分析による陳列最適化
  \item Webログ解析でのページ遷移パターン抽出
  \item レコメンデーションシステムの基盤
\end{itemize}

\subsection{利点}
\begin{itemize}
  \item 手軽に意思決定ルールを発見できる
  \item 可視化しやすくビジネス部門との連携に適す
\end{itemize}

\subsection{課題}
\begin{itemize}
  \item ルール数の爆発的増加に対処が必要
  \item 当たり前のルール(例:パン$\Rightarrow$牛乳)が大量に出力される
  \item 意外性の高いルールを見つけるにはリフトなどの指標でフィルタリングが必須
\end{itemize}

\section{おわりに}

相関ルールマイニングは,シンプルな指標と反復的探索によりデータ内の潜在的パターンを発見する有力手法である.一方,候補数の爆発や自明なルールの多発といった課題もあるため,その他の手法と組み合わせることで,実用的な知見を抽出できる.

\begin{thebibliography}{99}
  \bibitem{keio} DM08-04: Association Rules, Keio University (2008)\\
        \url{https://web.sfc.keio.ac.jp/~maunz/DM08/DM08-04.pdf}
\end{thebibliography}

\end{document} 