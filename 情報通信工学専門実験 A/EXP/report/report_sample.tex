\documentclass{report}
\usepackage[utf8]{inputenc}
\usepackage{CJKutf8}
\usepackage{amsmath}
\usepackage{amssymb}
\usepackage{amsfonts}
\usepackage{cite}
\usepackage{url}

% 印刷位置調整 %
% 必要に応じて値を変更してください.
\hoffset -10mm % <-- 左に 10mm 移動
\voffset -10mm % <-- 上に 10mm 移動

\newcommand{\AmSLaTeX}{%
 $\mathcal A$\lower.4ex\hbox{$\!\mathcal M\!$}$\mathcal S$-\LaTeX}
\newcommand{\PS}{{\scshape Post\-Script}}
\def\BibTeX{{\rmfamily B\kern-.05em{\scshape i\kern-.025em b}\kern-.08em
 T\kern-.1667em\lower.7ex\hbox{E}\kern-.125em X}}
\papernumber{姓 名  学籍番号}

\jtitle{レポートタイトル}
%\jsubtitle{サブタイトル} <- サブタイトルを付けたいときはこの行の先頭の % を取る
\authorlist{%
 \authorentry[youremailaddress@osaka-u.ac.jp]{姓 名}{your name}{osaka}% 
}
\affiliate[osaka]{学籍番号}
 {  Osaka University\\
}

%\MailAddress{$\dagger$hanako@deim.ac.jp,
% $\dagger\dagger$\{taro,jiro\}@jforum.co.jp}

\begin{document}

\pagestyle{empty}
\begin{jabstract}
今回の演習に関しての概要を書いてください.
\end{jabstract}

\maketitle

\section{はじめに}

\section{実装}

実装に関して適当に説明してください(配布資料に書いていることを写さないように).

\subsection{K-means}

\subsection{LOF}

\section{課題}

引用文献を参照する場合は,\cite{test1}で引用できます.

\subsection{課題番号X}

\subsection{課題番号Y}

\subsection{課題番号Z}

\section{おわりに}
なんか授業の感想や,不明な点,改善点とか書いてくれると有難いです.

\begin{thebibliography}{99}
\bibitem{test1} 引用文献1.
\bibitem{test2}引用文献2.
\end{thebibliography}



\end{document}
