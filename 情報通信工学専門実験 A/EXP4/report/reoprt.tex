\documentclass[12pt,a4paper]{article}
\usepackage[utf8]{inputenc}
\usepackage[japanese]{babel}
\usepackage{circuitikz}
\usepackage{tikz}
\usepackage{amsmath}
\usepackage{amsfonts}
\usepackage{geometry}

\geometry{margin=2.5cm}

\title{論理回路の図示例}
\author{情報通信工学専門実験A}
\date{\today}

\begin{document}

\maketitle

\section{ANDゲート}

\subsection{回路図}
\begin{center}
\begin{circuitikz}[scale=1.2]
    \draw (0,0) node[and port] (and1) {};
    \draw (and1.in 1) node[left] {A};
    \draw (and1.in 2) node[left] {B};
    \draw (and1.out) node[right] {Y = A $\cdot$ B};
\end{circuitikz}
\end{center}

\subsection{動作説明}
ANDゲートは2つの入力AとBがともに1(真)の場合のみ、出力Yが1となる論理ゲートです。
論理式は Y = A $\cdot$ B で表されます。

\subsection{真理値表}
\begin{center}
\begin{tabular}{|c|c|c|}
\hline
A & B & Y = A $\cdot$ B \\
\hline
0 & 0 & 0 \\
0 & 1 & 0 \\
1 & 0 & 0 \\
1 & 1 & 1 \\
\hline
\end{tabular}
\end{center}

\section{TeXでの図示方法}

\begin{itemize}
    \item \texttt{circuitikz}パッケージを使用:\texttt{\textbackslash usepackage\{circuitikz\}}
    \item ANDゲートの記述:\texttt{node[and port]}
    \item スケール調整:\texttt{[scale=1.2]}でサイズ調整
    \item 入力ラベル:\texttt{node[left] \{A\}}
    \item 出力ラベル:\texttt{node[right] \{Y\}}
\end{itemize}

\end{document}
