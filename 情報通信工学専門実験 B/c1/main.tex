\documentclass[a4paper,11pt]{jsarticle}

\usepackage{plautopatch}
\usepackage{amsmath}
\usepackage{amsfonts}
\usepackage{amssymb}
\usepackage[dvipdfmx]{graphicx}
\usepackage[dvipdfmx]{color} % dvipdfmxドライバーを明示的に指定
\DeclareGraphicsExtensions{.pdf,.png,.jpg}
\usepackage{xurl}
\usepackage{url}
\urlstyle{tt}
\usepackage{bm}
\usepackage{listings}
\usepackage{color}
\usepackage{float}
\usepackage[justification=centering]{caption}
\usepackage{fancyvrb}
\usepackage{fvextra}

% ページ設定
\usepackage{geometry}
\geometry{margin=25mm}
\setlength{\parindent}{0pt}

\begin{document}

\section*{目的}
パケットキャプチャログの分析を通じて、トランスポート層とアプリケーション層における代表的な通信プロトコルであるUser Datagram Protocol(UDP)、Transmission Control Protocol(TCP)、ならびにHypertext Transfer Protocol(HTTP)の基本動作を理解。加えて、Webアプリケーションにおける輻輳現象の観察を通じて通信トラヒックの基礎を学習する。

\section{第一週:パケットキャプチャによる通信ログ分析}
パケットキャプチャを用いて、HTTPメッセージの送受信、ならびにそれに用いられるトランスポート層プロトコルTCPの動作を確認した。また、トランスポート層プロトコルUDPが、Domain Name System(DNS)によりURLをIPアドレスに変換する際に使用されることを確認した。

\subsection{実験環境}
ハードウェア:
 ・Raspberry Pi 3 Model B+
ソフトウェア:
・Raspberry Pi OS 5.10(オペレーティング・システム)
・Wireshark 2.6.20(パケットキャプチャソフトウェア)
・Curl 7.64.0(HTTPをサポートするコマンドラインツール)
・Mozilla Firefox 78.9.0esr(Webブラウザ)

\subsection{理論}
第1週で用いるシステムの模式図を図1に示す。以降では、手元で操作する端末(Raspberry Pi)を単にクライアントと呼ぶ。

\begin{figure}[H]
\centering
\includegraphics[width=0.8\textwidth]{images/図1.png}
\caption{第1週で用いるシステムの模式図}
\end{figure}

 コマンドラインツール「Curl」を使用してWebページを閲覧する際には以下のような手順で通信が行われている。

・ドメイン名をIPアドレスに変換
1. クライアントは入力されたURLのコンテンツを保持するWebサーバのIPアドレスをDNSサーバに問い合わせる(名前解決)。この通信にはUDPが使用される。
2. DNSサーバは1の問い合わせ結果をクライアントに返す。この通信も、1と同様にUDPが使用される。

・TCPコネクションの確立
3. クライアントは2で通知されたIPアドレスと、HTTPにおけるサーバ側ポート番号の組を宛先として、TCPコネクション開始要求(SYN)を送信する。
4. Webサーバは、3で送信されたSYNを受信し、それに対する確認応答(ACK)に加え、コネクション開始要求(SYN)をクライアントに送信する。
5. クライアントは4で送信されたSYN/ACKを受信し、それに対するACKをWebサーバに送信する。これによりクライアントとWebサーバの間にTCPコネクションが確立される。

・HTTPメッセージの送受信
6. クライアントは5で確立されたTCPコネクションを通じて、Webサーバに「HTTP GETメッセージ」を格納したパケットを送信する。
7. Webサーバは、6で送信されたパケットを受信し、ACKをクライアントに返す。また、このパケットに含まれる「HTTP GETメッセージ」を読み込み、「HTTP Response メッセージ」をクライアントに返す。これにはクライアントが要求したWebページのソースコード(HTMLファイル)が含まれている。
8. クライアントは、7で送信された「HTTP Response メッセージ」を受信し、ACKを返す。

・TCPコネクションの終了
9. クライアントは、Webサーバにコネクション終了通知(FIN)を送信する。
10. Webサーバは、9で送信されたFINを受信し、それに対するACKに加え、コネクション終了通知(FIN)をクライアントに送信する。
11. クライアントは、10で送信されたFINを受信し、それに対するACKをWebサーバに送信する。

\subsection{実験課題1}
 この課題ではクライアントでターミナルを起動し、以下のコマンドを実行した。
\begin{Verbatim}[breaklines=true]
$ curl http://www2b.comm.eng.osaka-u.ac.jp/~yoshiaki/C1/page1.html
\end{Verbatim}

 実行結果は以下のようになった。
\begin{Verbatim}[breaklines=true]
ubuntu@raspberrypi3:~ $ curl http://www2b.comm.eng.osaka-u.ac.jp/~yoshiaki/C1/page1.html
<!DOCTYPE HTML PUBLIC "-//W3C//DTD HTML 4.01 Transitional//EN" "http://www.w3.org/TR/html4/loose.dtd">
<html lang="en">
<head>
<meta http-equiv="Content-Type" content="text/html"; charset="utf-8" />
<title>Page 1</title>
</head>
<body>
Hello, World!
</body>
</html>
\end{Verbatim}
ターミナルにWebページのソースコードが表示されていることがわかる。

\subsection{実験課題2}
 以下に示す手順で、実験課題1のコマンドを入力した時に行われた一連の通信に関するログを取得した。

1. クライアントでWiresharkを起動し、パケットキャプチャを開始。
2. 実験課題1と同じコマンドを実行
3. Webページのソースコードが標準出力に表示されたら、パケットキャプチャを停止。

 実行結果は以下のようになった。
\begin{Verbatim}[breaklines=true]
No.   Time     Source          Destination           Protocol Length Info
   8347 72.274886   192.168.12.66         10.144.0.1            DNS      88     Standard query 0x2cc8 A www2b.comm.eng.osaka-u.ac.jp
   8348 72.275140   192.168.12.66         10.144.0.1            DNS      88     Standard query 0x39c8 AAAA www2b.comm.eng.osaka-u.ac.jp
   8349 72.276049   10.144.0.1            192.168.12.66         DNS      104    Standard query response 0x2cc8 A www2b.comm.eng.osaka-u.ac.jp A 192.168.13.80
   8350 72.276242   10.144.0.1            192.168.12.66         DNS      160    Standard query response 0x39c8 AAAA www2b.comm.eng.osaka-u.ac.jp CNAME genji.comm.eng.osaka-u.ac.jp SOA gene.comm.eng.osaka-u.ac.jp
   8351 72.277170   192.168.12.66         192.168.13.80         TCP      74     35016 → 80 [SYN] Seq=1007184416 Win=64240 Len=0 MSS=1460 \texttt{SACK\_PERM}=1 TSval=3536644626 TSecr=0 WS=128
   8352 72.277751   192.168.13.80         192.168.12.66         TCP      74     80 → 35016 [SYN, ACK] Seq=2277431442 Ack=1007184417 Win=65160 Len=0 MSS=1460 \texttt{SACK\_PERM}=1 TSval=1932884923 TSecr=3536644626 WS=128
   8353 72.277875   192.168.12.66         192.168.13.80         TCP      66     35016 → 80 [ACK] Seq=1007184417 Ack=2277431443 Win=64256 Len=0 TSval=3536644627 TSecr=1932884923
   8354 72.278158   192.168.12.66         192.168.13.80         HTTP     181    GET /~yoshiaki/C1/page1.html HTTP/1.1 
   8355 72.278572   192.168.13.80         192.168.12.66         TCP      66     80 → 35016 [ACK] Seq=2277431443 Ack=1007184532 Win=65152 Len=0 TSval=1932884924 TSecr=3536644627
   8356 72.279152   192.168.13.80         192.168.12.66         HTTP     584    HTTP/1.1 200 OK  (text/html)
   8357 72.279238   192.168.12.66         192.168.13.80         TCP      66     35016 → 80 [ACK] Seq=1007184532 Ack=2277431961 Win=64000 Len=0 TSval=3536644628 TSecr=1932884925
   8358 72.280046   192.168.12.66         192.168.13.80         TCP      66     35016 → 80 [FIN, ACK] Seq=1007184532 Ack=2277431961 Win=64128 Len=0 TSval=3536644629 TSecr=1932884925
   8359 72.280709   192.168.13.80         192.168.12.66         TCP      66     80 → 35016 [FIN, ACK] Seq=2277431961 Ack=1007184533 Win=65152 Len=0 TSval=1932884926 TSecr=3536644629
   8360 72.280870   192.168.12.66         192.168.13.80         TCP      66     35016 → 80 [ACK] Seq=1007184533 Ack=2277431962 Win=64128 Len=0 TSval=3536644630 TSecr=1932884926
\end{Verbatim}

\subsubsection{ドメイン名をIPアドレスに変換}
ログでこの手続きを行っているのは、以下の部分である。
\begin{Verbatim}[breaklines=true]
 8347 72.274886   192.168.12.66         10.144.0.1            DNS      88     Standard query 0x2cc8 A www2b.comm.eng.osaka-u.ac.jp
  8349 72.276049   10.144.0.1            192.168.12.66         DNS      104    Standard query response 0x2cc8 A www2b.comm.eng.osaka-u.ac.jp A 192.168.13.80
\end{Verbatim}
 このログから、送信元IPアドレス、受信元IPアドレスを読み取ることができる。

 次に、パケットキャプチャログの詳細を抜粋して示す。
・8347
\begin{Verbatim}[breaklines=true]
User Datagram Protocol, Src Port: 57683, Dst Port: 53
    Source Port: 57683
    Destination Port: 53
    Length: 54
(中略)
    Queries
        www2b.comm.eng.osaka-u.ac.jp: type A, class IN
            Name: www2b.comm.eng.osaka-u.ac.jp
            [Name Length: 28]
            [Label Count: 6]
            Type: A (Host Address) (1)
            Class: IN (0x0001)
\end{Verbatim}

・8349
\begin{Verbatim}[breaklines=true]
User Datagram Protocol, Src Port: 53, Dst Port: 57683
    Source Port: 53
    Destination Port: 57683
    Length: 70
(中略)
    Queries
        www2b.comm.eng.osaka-u.ac.jp: type A, class IN
            Name: www2b.comm.eng.osaka-u.ac.jp
            [Name Length: 28]
            [Label Count: 6]
            Type: A (Host Address) (1)
            Class: IN (0x0001)
    Answers
        www2b.comm.eng.osaka-u.ac.jp: type A, class IN, addr 192.168.13.80
            Name: www2b.comm.eng.osaka-u.ac.jp
            Type: A (Host Address) (1)
            Class: IN (0x0001)
            Time to live: 1 (1 second)
            Data length: 4
            Address: 192.168.13.80
\end{Verbatim}

分析結果を表1にまとめた。

\begin{table}[h]
\centering
\caption{ドメイン名をIPアドレスに変換する手続きの分析}
\begin{tabular}{|c|c|c|c|c|c|}
\hline
No. & 送信元IPアドレス & 受信先IPアドレス & 送信元ポート番号 & 受信先ポート番号 & UDPセグメント長 \\
\hline
8347 & 192.168.12.66 & 10.144.0.1 & 57683 & 53 & 54 \\
8349 & 10.144.0.1 & 192.168.12.66 & 53 & 57683 & 70 \\
\hline
\end{tabular}
\end{table}

アプリケーション層のメッセージの内容を以下に示す。
 8347 : www2b.comm.eng.osaka-u.ac.jpのWebページを保持するIPアドレスの問い合わせ
 8349 : www2b.comm.eng.osaka-u.ac.jpのWebページを保持するIPアドレスを送信
  以上より、この2つのUDPパケットの送受信によって、クライアントは特定のドメイン名二 対応するIPアドレスを取得することができることがわかる。

\subsubsection{TCPコネクションの確立}
 ログでこの手続きを行っているのは、以下の部分である。
\begin{Verbatim}[breaklines=true]
8351 72.277170   192.168.12.66         192.168.13.80         TCP      74     35016 → 80 [SYN] Seq=1007184416 Win=64240 Len=0 MSS=1460 \texttt{SACK\_PERM}=1 TSval=3536644626 TSecr=0 WS=128
8352 72.277751   192.168.13.80         192.168.12.66         TCP      74     80 → 35016 [SYN, ACK] Seq=2277431442 Ack=1007184417 Win=65160 Len=0 MSS=1460 \texttt{SACK\_PERM}=1 TSval=1932884923 TSecr=3536644626 WS=128
8353 72.277875   192.168.12.66         192.168.13.80         TCP      66     35016 → 80 [ACK] Seq=1007184417 Ack=2277431443 Win=64256 Len=0 TSval=3536644627 TSecr=1932884923
\end{Verbatim}

次に、No.8351のパケットキャプチャログの詳細を抜粋して示す。
\begin{Verbatim}[breaklines=true]
Transmission Control Protocol, Src Port: 35016, Dst Port: 80, Seq: 1007184416, Len: 0
(中略)
    Acknowledgment number: 0
\end{Verbatim}

ログから読み取ることができる情報を表2、表3にまとめる。
\begin{table}[h]
\centering
\caption{TCPコネクションを確立するパケットの分析1}
\begin{tabular}{|c|c|c|c|c|c|}
\hline
No. & 送信元IPアドレス & 受信先IPアドレス & 送信元ポート番号 & 受信先ポート番号 & TCPセグメント長 \\
\hline
8351 & 192.168.12.66 & 192.168.13.80 & 35016 & 80 & 0 \\
8352 & 192.168.13.80 & 192.168.12.66 & 80 & 35016 & 0 \\
8353 & 192.168.12.66 & 192.168.13.80 & 35016 & 80 & 0 \\
\hline
\end{tabular}
\end{table}

\begin{table}[h]
\centering
\caption{TCPコネクションを確立するパケットの分析2}
\begin{tabular}{|c|c|c|c|}
\hline
No. & ONになっているTCPフラグ & シーケンス番号 & ACK番号 \\
\hline
8351 & SYN & 1007184416 & 0 \\
8352 & SYN, ACK & 2277431442 & 1007184417 \\
8353 & ACK & 1007184417 & 2277431443 \\
\hline
\end{tabular}
\end{table}

 No.8351において、クライアントがSYNを送信し、No.8352のACK番号は、No.8351のシーケンス番号に1を加えたACKとSYNが返されている。同様に、No.8353ではNo.8352のACK番号と同じシーケンス番号のACKが返されており、これら3つのパケットでTCP接続の確立が行われたことがわかる。なお、この3つのパケットにはアプリケーション層のメッセージが存在しない。

\subsubsection{HTTPメッセージの送受信}
 ログでこの手続きを行っているのは、以下の部分である。
\begin{Verbatim}[breaklines=true]
8354 72.278158   192.168.12.66         192.168.13.80         HTTP     181    GET /~yoshiaki/C1/page1.html HTTP/1.1 

8355 72.278572   192.168.13.80         192.168.12.66         TCP      66     80 → 35016 [ACK] Seq=2277431443 Ack=1007184532 Win=65152 Len=0 TSval=1932884924 TSecr=3536644627

8356 72.279152   192.168.13.80         192.168.12.66         HTTP     584    HTTP/1.1 200 OK  (text/html)

8357 72.279238   192.168.12.66         192.168.13.80         TCP      66     35016 → 80 [ACK] Seq=1007184532 Ack=2277431961 Win=64000 Len=0 TSval=3536644628 TSecr=1932884925
\end{Verbatim}

次に、それぞれのパケットキャプチャログの詳細を抜粋して示す。
・8354
\begin{Verbatim}[breaklines=true]
Transmission Control Protocol, Src Port: 35016, Dst Port: 80, Seq: 1007184417, Ack: 2277431443, Len: 115
(中略)
    Flags: 0x018 (PSH, ACK)
   (中略)
Hypertext Transfer Protocol
    GET /~yoshiaki/C1/page1.html HTTP/1.1\textbackslash r\textbackslash n
        [Expert Info (Chat/Sequence): GET /~yoshiaki/C1/page1.html HTTP/1.1\textbackslash r\textbackslash n]
            [GET /~yoshiaki/C1/page1.html HTTP/1.1\textbackslash r\textbackslash n]
            [Severity level: Chat]
            [Group: Sequence]
        Request Method: GET
        Request URI: /~yoshiaki/C1/page1.html
        Request Version: HTTP/1.1
    Host: www2b.comm.eng.osaka-u.ac.jp\textbackslash r\textbackslash n
    User-Agent: curl/7.64.0\textbackslash r\textbackslash n
    Accept: */*\textbackslash r\textbackslash n
    \textbackslash r\textbackslash n
    [Full request URI: http://www2b.comm.eng.osaka-u.ac.jp/~yoshiaki/C1/page1.html]
    [HTTP request 1/1]
    [Response in frame: 8356]
\end{Verbatim}


・8356
\begin{Verbatim}[breaklines=true]
Transmission Control Protocol, Src Port: 80, Dst Port: 35016, Seq: 2277431443, Ack: 1007184532, Len: 518
(中略)
    Flags: 0x018 (PSH, ACK)
(中略)
Hypertext Transfer Protocol
    HTTP/1.1 200 OK\textbackslash r\textbackslash n
        [Expert Info (Chat/Sequence): HTTP/1.1 200 OK\textbackslash r\textbackslash n]
            [HTTP/1.1 200 OK\textbackslash r\textbackslash n]
            [Severity level: Chat]
            [Group: Sequence]
        Response Version: HTTP/1.1
        Status Code: 200
        [Status Code Description: OK]
        Response Phrase: OK
    Date: Tue, 11 May 2021 04:04:01 GMT\textbackslash r\textbackslash n
    Server: Apache/2.4.41 (Ubuntu)\textbackslash r\textbackslash n
    Last-Modified: Thu, 17 Sep 2020 03:12:39 GMT\textbackslash r\textbackslash n
    ETag: "10a-5af79c1bb058b"\textbackslash r\textbackslash n
    Accept-Ranges: bytes\textbackslash r\textbackslash n
    Content-Length: 266\textbackslash r\textbackslash n
        [Content length: 266]
    Vary: Accept-Encoding\textbackslash r\textbackslash n
    Content-Type: text/html\textbackslash r\textbackslash n
    \textbackslash r\textbackslash n
    [HTTP response 1/1]
    [Time since request: 0.000993430 seconds]
    [Request in frame: 8354]
    [Request URI: http://www2b.comm.eng.osaka-u.ac.jp/~yoshiaki/C1/page1.html]
    File Data: 266 bytes
\end{Verbatim}

 このログから、それぞれのパケットにおける送信元ポート番号、受信先ポート番号、TCPセグメント長、シーケンス番号、ACK番号、ONになっているTCPフラグを読み取ることができる。各パケットの情報を表4と表5にまとめる。
\begin{table}[h]
\centering
\caption{HTTPメッセージの送受信の手続きの分析1}
\begin{tabular}{|c|c|c|c|c|c|}
\hline
No. & 送信元IPアドレス & 受信先IPアドレス & 送信元ポート番号 & 受信先ポート番号 & TCPセグメント長 \\
\hline
8354 & 192.168.12.66 & 192.168.13.80 & 35016 & 80 & 115 \\
8355 & 192.168.13.80 & 192.168.12.66 & 80 & 35016 & 0 \\
8356 & 192.168.13.80 & 192.168.12.66 & 80 & 35016 & 518 \\
8357 & 192.168.12.66 & 192.168.13.80 & 35016 & 80 & 0 \\
\hline
\end{tabular}
\end{table}

\begin{table}[h]
\centering
\caption{HTTPメッセージの送受信の手続きの分析2}
\begin{tabular}{|c|c|c|c|}
\hline
No. & ONになっているTCPフラグ & シーケンス番号 & ACK番号 \\
\hline
8354 & PSH, ACK & 1007184417 & 2277431443 \\
8355 & ACK & 2277431443 & 1007184532 \\
8356 & PSH, ACK & 2277431443 & 1007184532 \\
8357 & ACK & 1007184532 & 2277431961 \\
\hline
\end{tabular}
\end{table}

アプリケーション層のメッセージの内容を以下に示す。
 8354 : Webサーバに「HTTP GETメッセージ」を送信して
    /~yoshiaki/C1/page1.htmlのページを要求
 8355 : なし
 8356 : 「HTTP GET メッセージ」に対する「HTTP Response メッセージ」を送信
 8357 : なし

 表5からNo.8355のシーケンス番号とNo.8354のACK番号が等しいことから、これはNo.8354の「HTTP GETメッセージ」に対するACKであることが確認できる。同様に、No.8357のパケットのシーケンス番号とNo.8356のパケットのACK番号が等しいため、No.8356の「HTTP Responseメッセージ」に対するACKであることがわかる。

\subsubsection{TCPコネクションの終了}
 ログでこの手続きを行っているのは、以下の部分である。
\begin{Verbatim}[breaklines=true]
 8358 72.280046   192.168.12.66         192.168.13.80         TCP      66     35016 → 80 [FIN, ACK] Seq=1007184532 Ack=2277431961 Win=64128 Len=0 TSval=3536644629 TSecr=1932884925
 8359 72.280709   192.168.13.80         192.168.12.66         TCP      66     80 → 35016 [FIN, ACK] Seq=2277431961 Ack=1007184533 Win=65152 Len=0 TSval=1932884926 TSecr=3536644629
 8360 72.280870   192.168.12.66         192.168.13.80         TCP      66     35016 → 80 [ACK] Seq=1007184533 Ack=2277431962 Win=64128 Len=0 TSval=3536644630 TSecr=1932884926
\end{Verbatim}

 このログから読み取ることができる各パケットの情報を表6と表7にまとめる。
\begin{table}[h]
\centering
\caption{TCPコネクションを終了する手続きの分析1}
\begin{tabular}{|c|c|c|c|c|c|}
\hline
No. & 送信元IPアドレス & 受信先IPアドレス & 送信元ポート番号 & 受信先ポート番号 & TCPセグメント長 \\
\hline
8358 & 192.168.12.66 & 192.168.13.80 & 35016 & 80 & 0 \\
8359 & 192.168.13.80 & 192.168.12.66 & 80 & 35016 & 0 \\
8360 & 192.168.12.66 & 192.168.13.80 & 35016 & 80 & 0 \\
\hline
\end{tabular}
\end{table}

\begin{table}[h]
\centering
\caption{TCPコネクションを終了する手続きの分析2}
\begin{tabular}{|c|c|c|c|}
\hline
No. & ONになっているTCPフラグ & シーケンス番号 & ACK番号 \\
\hline
8358 & ACK, FIN & 1007184532 & 2277431961 \\
8359 & ACK, FIN & 2277431961 & 1007184533 \\
8360 & ACK & 1007184533 & 2277431962 \\
\hline
\end{tabular}
\end{table}

 No.8358のパケットでは、FINをWebサーバに送信し、No.8359でそのACKが返されている。同様に、No.8360でもNo.8359のFINに対するACK番号が送信されており、コネクションの終了処理が進んでいることがわかる。また、この3つのパケットにはいずれもアプリケーション層のメッセージは存在しない。

\subsection{実験課題3}
 この課題では、実験課題2と同様の手順でWebブラウザ(Firefox)から以下にアクセスした際の通信ログを分析する。
\url{http://www2b.comm.eng.osaka-u.ac.jp/~yoshiaki/C1/page1.html} 

取得したのは以下のログである。このログを分析し、結果をまとめる。
\begin{Verbatim}[breaklines=true]
No.     Time        Source                Destination           Protocol Length Info
    688 17.977035   192.168.12.66         10.144.0.1            DNS      84     Standard query 0x140f A detectportal.firefox.com
    689 17.979484   10.144.0.1            192.168.12.66         DNS      489    Standard query response 0x140f A detectportal.firefox.com CNAME detectportal.prod.mozaws.net CNAME prod.detectportal.prod.cloudops.mozgcp.net A 34.107.221.82 NS ns-cloud-e4.googledomains.com NS ns-cloud-e1.googledomains.com NS ns-cloud-e3.googledomains.com NS ns-cloud-e2.googledomains.com A 216.239.32.110 A 216.239.34.110 A 216.239.36.110 A 216.239.38.110 AAAA 2001:4860:4802:32::6e AAAA 2001:4860:4802:34::6e AAAA 2001:4860:4802:36::6e AAAA 2001:4860:4802:38::6e
    710 18.273600   192.168.12.66         10.144.0.1            DNS      71     Standard query 0xbc85 A mozilla.org
    711 18.274560   10.144.0.1            192.168.12.66         DNS      304    Standard query response 0xbc85 A mozilla.org A 44.236.48.31 A 44.235.246.155 A 44.236.72.93 NS ns7-66.akam.net NS ns1-240.akam.net NS ns4-64.akam.net NS ns5-65.akam.net A 184.85.248.65 A 193.108.91.240 A 84.53.139.64 A 96.7.49.66 AAAA 2600:1401:2::f0
    712 18.274845   192.168.12.66         10.144.0.1            DNS      71     Standard query 0x3aba A mozilla.org
    714 18.275518   10.144.0.1            192.168.12.66         DNS      304    Standard query response 0x3aba A mozilla.org A 44.236.48.31 A 44.235.246.155 A 44.236.72.93 NS ns7-66.akam.net NS ns1-240.akam.net NS ns4-64.akam.net NS ns5-65.akam.net A 184.85.248.65 A 193.108.91.240 A 84.53.139.64 A 96.7.49.66 AAAA 2600:1401:2::f0
    715 18.276111   192.168.12.66         10.144.0.1            DNS      84     Standard query 0xa2bf A detectportal.firefox.com
    716 18.276852   10.144.0.1            192.168.12.66         DNS      489    Standard query response 0xa2bf A detectportal.firefox.com CNAME detectportal.prod.mozaws.net CNAME prod.detectportal.prod.cloudops.mozgcp.net A 34.107.221.82 NS ns-cloud-e4.googledomains.com NS ns-cloud-e1.googledomains.com NS ns-cloud-e3.googledomains.com NS ns-cloud-e2.googledomains.com A 216.239.32.110 A 216.239.34.110 A 216.239.36.110 A 216.239.38.110 AAAA 2001:4860:4802:32::6e AAAA 2001:4860:4802:34::6e AAAA 2001:4860:4802:36::6e AAAA 2001:4860:4802:38::6e
    717 18.277491   192.168.12.66         10.144.0.1            DNS      84     Standard query 0xfdb4 A detectportal.firefox.com
    720 18.278186   10.144.0.1            192.168.12.66         DNS      489    Standard query response 0xfdb4 A detectportal.firefox.com CNAME detectportal.prod.mozaws.net CNAME prod.detectportal.prod.cloudops.mozgcp.net A 34.107.221.82 NS ns-cloud-e4.googledomains.com NS ns-cloud-e1.googledomains.com NS ns-cloud-e3.googledomains.com NS ns-cloud-e2.googledomains.com A 216.239.32.110 A 216.239.34.110 A 216.239.36.110 A 216.239.38.110 AAAA 2001:4860:4802:32::6e AAAA 2001:4860:4802:34::6e AAAA 2001:4860:4802:36::6e AAAA 2001:4860:4802:38::6e
    743 18.901046   192.168.12.66         10.144.0.1            DNS      95     Standard query 0x2777 A content-signature-2.cdn.mozilla.net
    744 18.903489   10.144.0.1            192.168.12.66         DNS      368    Standard query response 0x2777 A content-signature-2.cdn.mozilla.net CNAME d2nxq2uap88usk.cloudfront.net A 13.33.29.20 A 13.33.29.2 A 13.33.29.106 A 13.33.29.125 NS ns-1295.awsdns-33.org NS ns-365.awsdns-45.com NS ns-704.awsdns-24.net NS ns-1811.awsdns-34.co.uk A 205.251.197.15 A 205.251.194.192
    779 19.077282   192.168.12.66         10.144.0.1            DNS      97     Standard query 0xb487 A firefox.settings.services.mozilla.com
    781 19.079781   10.144.0.1            192.168.12.66         DNS      330    Standard query response 0xb487 A firefox.settings.services.mozilla.com A 99.86.193.95 A 99.86.193.125 A 99.86.193.83 A 99.86.193.91 NS ns-1364.awsdns-42.org NS ns-1627.awsdns-11.co.uk NS ns-166.awsdns-20.com NS ns-972.awsdns-57.net A 205.251.192.166 A 205.251.195.204
   1018 20.715609   192.168.12.66         10.144.0.1            DNS      85     Standard query 0xe902 A push.services.mozilla.com
   1019 20.718021   10.144.0.1            192.168.12.66         DNS      305    Standard query response 0xe902 A push.services.mozilla.com CNAME autopush.prod.mozaws.net A 35.163.208.27 NS ns-614.awsdns-12.net NS ns-377.awsdns-47.com NS ns-1260.awsdns-29.org NS ns-1986.awsdns-56.co.uk A 205.251.196.236 A 205.251.194.102
   1023 20.770565   192.168.12.66         10.144.0.1            DNS      85     Standard query 0xc06b A push.services.mozilla.com
   1024 20.771246   10.144.0.1            192.168.12.66         DNS      305    Standard query response 0xc06b A push.services.mozilla.com CNAME autopush.prod.mozaws.net A 35.163.208.27 NS ns-614.awsdns-12.net NS ns-377.awsdns-47.com NS ns-1260.awsdns-29.org NS ns-1986.awsdns-56.co.uk A 205.251.196.236 A 205.251.194.102
   1025 20.772946   192.168.12.66         10.144.0.1            DNS      85     Standard query 0x0761 A push.services.mozilla.com
   1026 20.773595   10.144.0.1            192.168.12.66         DNS      305    Standard query response 0x0761 A push.services.mozilla.com CNAME autopush.prod.mozaws.net A 35.163.208.27 NS ns-614.awsdns-12.net NS ns-377.awsdns-47.com NS ns-1260.awsdns-29.org NS ns-1986.awsdns-56.co.uk A 205.251.196.236 A 205.251.194.102
   1076 21.098356   192.168.12.66         10.144.0.1            DNS      77     Standard query 0xfe6b A ocsp.digicert.com
   1077 21.099169   10.144.0.1            192.168.12.66         DNS      373    Standard query response 0xfe6b A ocsp.digicert.com CNAME cs9.wac.phicdn.net A 117.18.237.29 NS ns3.phicdn.net NS ns1.phicdn.net NS ns4.phicdn.net NS ns2.phicdn.net A 72.21.80.5 A 72.21.80.6 A 192.229.254.5 A 192.229.254.6 AAAA 2606:2800:1::5 AAAA 2606:2800:1::6 AAAA 2606:2800:e::5 AAAA 2606:2800:e::6
   2622 34.946600   192.168.12.66         10.144.0.1            DNS      88     Standard query 0x2089 A www2b.comm.eng.osaka-u.ac.jp
   2623 34.947637   10.144.0.1            192.168.12.66         DNS      104    Standard query response 0x2089 A www2b.comm.eng.osaka-u.ac.jp A 192.168.13.80
   2639 35.055745   192.168.12.66         192.168.13.80         TCP      74     35516 → 80 [SYN] Seq=3906406002 Win=64240 Len=0 MSS=1460 \texttt{SACK\_PERM}=1 TSval=3543356595 TSecr=0 WS=128
   2640 35.056306   192.168.13.80         192.168.12.66         TCP      74     80 → 35516 [SYN, ACK] Seq=35308251 Ack=3906406003 Win=65160 Len=0 MSS=1460 \texttt{SACK\_PERM}=1 TSval=1939596895 TSecr=3543356595 WS=128
   2641 35.056438   192.168.12.66         192.168.13.80         TCP      66     35516 → 80 [ACK] Seq=3906406003 Ack=35308252 Win=64256 Len=0 TSval=3543356596 TSecr=1939596895
   2642 35.056808   192.168.12.66         192.168.13.80         HTTP     436    GET /~yoshiaki/C1/page1.html HTTP/1.1 
   2643 35.057355   192.168.13.80         192.168.12.66         TCP      66     80 → 35516 [ACK] Seq=35308252 Ack=3906406373 Win=64896 Len=0 TSval=1939596896 TSecr=3543356597
   2644 35.058177   192.168.13.80         192.168.12.66         HTTP     627    HTTP/1.1 200 OK  (text/html)
   2645 35.058247   192.168.12.66         192.168.13.80         TCP      66     35516 → 80 [ACK] Seq=3906406373 Ack=35308813 Win=64000 Len=0 TSval=3543356598 TSecr=1939596897
   2707 35.533754   192.168.12.66         192.168.13.80         HTTP     334    GET /favicon.ico HTTP/1.1 
   2708 35.534299   192.168.13.80         192.168.12.66         TCP      66     80 → 35516 [ACK] Seq=35308813 Ack=3906406641 Win=64640 Len=0 TSval=1939597373 TSecr=3543357073
   2709 35.534821   192.168.13.80         192.168.12.66         HTTP     572    HTTP/1.1 404 Not Found  (text/html)
   2710 35.534867   192.168.12.66         192.168.13.80         TCP      66     35516 → 80 [ACK] Seq=3906406641 Ack=35309319 Win=64000 Len=0 TSval=3543357075 TSecr=1939597373
   3222 40.534947   192.168.13.80         192.168.12.66         TCP      66     80 → 35516 [FIN, ACK] Seq=35309319 Ack=3906406641 Win=64640 Len=0 TSval=1939602374 TSecr=3543357075
   3223 40.535294   192.168.12.66         192.168.13.80         TCP      66     35516 → 80 [FIN, ACK] Seq=3906406641 Ack=35309320 Win=64128 Len=0 TSval=3543362075 TSecr=1939602374
   3224 40.535752   192.168.13.80         192.168.12.66         TCP      66     80 → 35516 [ACK] Seq=35309320 Ack=3906406642 Win=64640 Len=0 TSval=1939602374 TSecr=3543362075
\end{Verbatim}

\subsubsection{ドメイン名をIPアドレスに変換}
ログでこの手続きを行っているのは、以下の部分である。
\begin{Verbatim}[breaklines=true]
  2622 34.946600   192.168.12.66         10.144.0.1            DNS      88     Standard query 0x2089 A www2b.comm.eng.osaka-u.ac.jp
  2623 34.947637   10.144.0.1            192.168.12.66         DNS      104    Standard query response 0x2089 A www2b.comm.eng.osaka-u.ac.jp A 192.168.13.80
\end{Verbatim}
 このログから、送信元IPアドレス、受信元IPアドレスを読み取ることができる。
 次に、No.2622のパケットキャプチャログの詳細を抜粋して示す。
\begin{Verbatim}[breaklines=true]
User Datagram Protocol, Src Port: 49142, Dst Port: 53
    Source Port: 49142
    Destination Port: 53
    Length: 54
(中略)
    Queries
        www2b.comm.eng.osaka-u.ac.jp: type A, class IN
            Name: www2b.comm.eng.osaka-u.ac.jp
            [Name Length: 28]
            [Label Count: 6]
            Type: A (Host Address) (1)
            Class: IN (0x0001)
\end{Verbatim}
 このログから送信元ポート番号、受信先ポート番号、UDPセグメント長を読み取ることができる。

 同様に、No.2623のパケットキャプチャログの詳細を抜粋して示す。
\begin{Verbatim}[breaklines=true]
User Datagram Protocol, Src Port: 53, Dst Port: 49142
    Source Port: 53
    Destination Port: 49142
    Length: 70
(中略)
    Queries
        www2b.comm.eng.osaka-u.ac.jp: type A, class IN
            Name: www2b.comm.eng.osaka-u.ac.jp
            [Name Length: 28]
            [Label Count: 6]
            Type: A (Host Address) (1)
            Class: IN (0x0001)
    Answers
        www2b.comm.eng.osaka-u.ac.jp: type A, class IN, addr 192.168.13.80
            Name: www2b.comm.eng.osaka-u.ac.jp
            Type: A (Host Address) (1)
            Class: IN (0x0001)
            Time to live: 1 (1 second)
            Data length: 4
            Address: 192.168.13.80
\end{Verbatim}

No.2623のパケットについてもログから送信元ポート番号、受信先ポート番号、UDPセグメント長を読み取ることができる。
 分析結果を表8にまとめる。
\begin{table}[h]
\centering
\caption{ドメイン名をIPアドレスに変換する手続きの分析}
\begin{tabular}{|c|c|c|c|c|c|}
\hline
No. & 送信元IPアドレス & 受信先IPアドレス & 送信元ポート番号 & 受信先ポート番号 & UDPセグメント長 \\
\hline
2622 & 192.168.12.66 & 10.144.0.1 & 49142 & 53 & 54 \\
2623 & 10.144.0.1 & 192.168.12.66 & 53 & 49142 & 70 \\
\hline
\end{tabular}
\end{table}
アプリケーション層のメッセージの内容を以下に示す。
 2622 : www2b.comm.eng.osaka-u.ac.jpのWebページを保持するIPアドレスの問い合わせ
 2623 : www2b.comm.eng.osaka-u.ac.jpのWebページを保持するIPアドレスを送信
 以上より、2つのUDPパケットの送受信によってクライアントは特定のドメイン名に対応するIPアドレスを取得できていることがわかる。

\subsubsection{TCPコネクションの確立}
 ログでこの手続きを行っているのは、以下の部分である。
\begin{Verbatim}[breaklines=true]
  2639 35.055745   192.168.12.66         192.168.13.80         TCP      74     35516 → 80 [SYN] Seq=3906406002 Win=64240 Len=0 MSS=1460 \texttt{SACK\_PERM}=1 TSval=3543356595 TSecr=0 WS=128
  2640 35.056306   192.168.13.80         192.168.12.66         TCP      74     80 → 35516 [SYN, ACK] Seq=35308251 Ack=3906406003 Win=65160 Len=0 MSS=1460 \texttt{SACK\_PERM}=1 TSval=1939596895 TSecr=3543356595 WS=128
  2641 35.056438   192.168.12.66         192.168.13.80         TCP      66     35516 → 80 [ACK] Seq=3906406003 Ack=35308252 Win=64256 Len=0 TSval=3543356596 TSecr=1939596895
\end{Verbatim}

次に、No.2639のパケットキャプチャログの詳細を抜粋して示す。
\begin{Verbatim}[breaklines=true]
Transmission Control Protocol, Src Port: 35516, Dst Port: 80, Seq: 3906406002, Len: 0
(中略)
Acknowledgment number: 0
\end{Verbatim}

ログから読み取ることができる情報を表9、表10にまとめる。
\begin{table}[h]
\centering
\caption{TCPコネクションを確立する手続きの分析1}
\begin{tabular}{|c|c|c|c|c|c|}
\hline
No. & 送信元IPアドレス & 受信先IPアドレス & 送信元ポート番号 & 受信先ポート番号 & TCPセグメント長 \\
\hline
2639 & 192.168.12.66 & 192.168.13.80 & 35516 & 80 & 0 \\
2640 & 192.168.13.80 & 192.168.12.66 & 80 & 35516 & 0 \\
2641 & 192.168.12.66 & 192.168.13.80 & 35516 & 80 & 0 \\
\hline
\end{tabular}
\end{table}

\begin{table}[h]
\centering
\caption{TCPコネクションを確立する手続きの分析2}
\begin{tabular}{|c|c|c|c|}
\hline
No. & ONになっているTCPフラグ & シーケンス番号 & ACK番号 \\
\hline
2639 & SYN & 3906406002 & 0 \\
2640 & SYN, ACK & 35308251 & 3906406003 \\
2641 & ACK & 3906406003 & 35308252 \\
\hline
\end{tabular}
\end{table}

 以上より、実験課題2のNo.8351からNo.8353のパケットと同じ役割であることがわかる。

\subsubsection{HTTPメッセージの送受信}
 ログでこの手続きを行っているのは、以下の部分である。
\begin{Verbatim}[breaklines=true]
 2642 35.056808   192.168.12.66         192.168.13.80         HTTP     436    GET /~yoshiaki/C1/page1.html HTTP/1.1 
  2643 35.057355   192.168.13.80         192.168.12.66         TCP      66     80 → 35516 [ACK] Seq=35308252 Ack=3906406373 Win=64896 Len=0 TSval=1939596896 TSecr=3543356597
  2644 35.058177   192.168.13.80         192.168.12.66         HTTP     627    HTTP/1.1 200 OK  (text/html)
  2645 35.058247   192.168.12.66         192.168.13.80         TCP      66     35516 → 80 [ACK] Seq=3906406373 Ack=35308813 Win=64000 Len=0 TSval=3543356598 TSecr=1939596897
  2707 35.533754   192.168.12.66         192.168.13.80         HTTP     334    GET /favicon.ico HTTP/1.1 
  2708 35.534299   192.168.13.80         192.168.12.66         TCP      66     80 → 35516 [ACK] Seq=35308813 Ack=3906406641 Win=64640 Len=0 TSval=1939597373 TSecr=3543357073
  2709 35.534821   192.168.13.80         192.168.12.66         HTTP     572    HTTP/1.1 404 Not Found  (text/html)
  2710 35.534867   192.168.12.66         192.168.13.80         TCP      66     35516 → 80 [ACK] Seq=3906406641 Ack=35309319 Win=64000 Len=0 TSval=3543357075 TSecr=1939597373
\end{Verbatim}

 次に、それぞれのパケットキャプチャログの詳細を抜粋して示す。
・2642
\begin{Verbatim}[breaklines=true]
Transmission Control Protocol, Src Port: 35516, Dst Port: 80, Seq: 3906406003, Ack: 35308252, Len: 370
(中略)
    Flags: 0x018 (PSH, ACK)
(中略)
Hypertext Transfer Protocol
    GET /~yoshiaki/C1/page1.html HTTP/1.1\textbackslash r\textbackslash n
        [Expert Info (Chat/Sequence): GET /~yoshiaki/C1/page1.html HTTP/1.1\textbackslash r\textbackslash n]
            [GET /~yoshiaki/C1/page1.html HTTP/1.1\textbackslash r\textbackslash n]
            [Severity level: Chat]
            [Group: Sequence]
        Request Method: GET
        Request URI: /~yoshiaki/C1/page1.html
        Request Version: HTTP/1.1
    Host: www2b.comm.eng.osaka-u.ac.jp\textbackslash r\textbackslash n
    User-Agent: Mozilla/5.0 (X11; Linux armv7l; rv:78.0) Gecko/20100101 Firefox/78.0\textbackslash r\textbackslash n
    Accept: text/html,application/xhtml+xml,application/xml;q=0.9,image/webp,*/*;q=0.8\textbackslash r\textbackslash n
    Accept-Language: en-US,en;q=0.5\textbackslash r\textbackslash n
    Accept-Encoding: gzip, deflate\textbackslash r\textbackslash n
    DNT: 1\textbackslash r\textbackslash n
    Connection: keep-alive\textbackslash r\textbackslash n
    Upgrade-Insecure-Requests: 1\textbackslash r\textbackslash n
    \textbackslash r\textbackslash n
    [Full request URI: http://www2b.comm.eng.osaka-u.ac.jp/~yoshiaki/C1/page1.html]
    [HTTP request 1/2]
    [Response in frame: 2644]
    [Next request in frame: 2707]

・2644
Transmission Control Protocol, Src Port: 80, Dst Port: 35516, Seq: 35308252, Ack: 3906406373, Len: 561
(中略)
    Flags: 0x018 (PSH, ACK)
(中略)
Hypertext Transfer Protocol
    HTTP/1.1 200 OK\textbackslash r\textbackslash n
        [Expert Info (Chat/Sequence): HTTP/1.1 200 OK\textbackslash r\textbackslash n]
            [HTTP/1.1 200 OK\textbackslash r\textbackslash n]
            [Severity level: Chat]
            [Group: Sequence]
        Response Version: HTTP/1.1
        Status Code: 200
        [Status Code Description: OK]
        Response Phrase: OK
    Date: Tue, 11 May 2021 05:55:53 GMT\textbackslash r\textbackslash n
    Server: Apache/2.4.41 (Ubuntu)\textbackslash r\textbackslash n
    Last-Modified: Thu, 17 Sep 2020 03:12:39 GMT\textbackslash r\textbackslash n
    ETag: "10a-5af79c1bb058b-gzip"\textbackslash r\textbackslash n
    Accept-Ranges: bytes\textbackslash r\textbackslash n
    Vary: Accept-Encoding\textbackslash r\textbackslash n
    Content-Encoding: gzip\textbackslash r\textbackslash n
    Content-Length: 224\textbackslash r\textbackslash n
        [Content length: 224]
    Keep-Alive: timeout=5, max=100\textbackslash r\textbackslash n
    Connection: Keep-Alive\textbackslash r\textbackslash n
    Content-Type: text/html\textbackslash r\textbackslash n
    \textbackslash r\textbackslash n
    [HTTP response 1/2]
    [Time since request: 0.001369270 seconds]
    [Request in frame: 2642]
    [Next request in frame: 2707]
    [Next response in frame: 2709]
    [Request URI: http://www2b.comm.eng.osaka-u.ac.jp/~yoshiaki/C1/page1.html]
    Content-encoded entity body (gzip): 224 bytes -> 266 bytes
    File Data: 266 bytes
\end{Verbatim}

・2707
\begin{Verbatim}[breaklines=true]
Transmission Control Protocol, Src Port: 35516, Dst Port: 80, Seq: 3906406373, Ack: 35308813, Len: 268
(中略)
    Flags: 0x018 (PSH, ACK)
(中略)
Hypertext Transfer Protocol
    GET /favicon.ico HTTP/1.1\textbackslash r\textbackslash n
        [Expert Info (Chat/Sequence): GET /favicon.ico HTTP/1.1\textbackslash r\textbackslash n]
            [GET /favicon.ico HTTP/1.1\textbackslash r\textbackslash n]
            [Severity level: Chat]
            [Group: Sequence]
        Request Method: GET
        Request URI: /favicon.ico
        Request Version: HTTP/1.1
    Host: www2b.comm.eng.osaka-u.ac.jp\textbackslash r\textbackslash n
    User-Agent: Mozilla/5.0 (X11; Linux armv7l; rv:78.0) Gecko/20100101 Firefox/78.0\textbackslash r\textbackslash n
    Accept: image/webp,*/*\textbackslash r\textbackslash n
    Accept-Language: en-US,en;q=0.5\textbackslash r\textbackslash n
    Accept-Encoding: gzip, deflate\textbackslash r\textbackslash n
    DNT: 1\textbackslash r\textbackslash n
    Connection: keep-alive\textbackslash r\textbackslash n
    \textbackslash r\textbackslash n
    [Full request URI: http://www2b.comm.eng.osaka-u.ac.jp/favicon.ico]
    [HTTP request 2/2]
    [Prev request in frame: 2642]
    [Response in frame: 2709]

・2709
\begin{Verbatim}[breaklines=true]
Transmission Control Protocol, Src Port: 80, Dst Port: 35516, Seq: 35308813, Ack: 3906406641, Len: 506
(中略)
    Flags: 0x018 (PSH, ACK)
(中略)
Hypertext Transfer Protocol
    HTTP/1.1 404 Not Found\textbackslash r\textbackslash n
        [Expert Info (Chat/Sequence): HTTP/1.1 404 Not Found\textbackslash r\textbackslash n]
            [HTTP/1.1 404 Not Found\textbackslash r\textbackslash n]
            [Severity level: Chat]
            [Group: Sequence]
        Response Version: HTTP/1.1
        Status Code: 404
        [Status Code Description: Not Found]
        Response Phrase: Not Found
    Date: Tue, 11 May 2021 05:55:54 GMT\textbackslash r\textbackslash n
    Server: Apache/2.4.41 (Ubuntu)\textbackslash r\textbackslash n
    Content-Length: 290\textbackslash r\textbackslash n
        [Content length: 290]
    Keep-Alive: timeout=5, max=99\textbackslash r\textbackslash n
    Connection: Keep-Alive\textbackslash r\textbackslash n
    Content-Type: text/html; charset=iso-8859-1\textbackslash r\textbackslash n
    \textbackslash r\textbackslash n
    [HTTP response 2/2]
    [Time since request: 0.001067291 seconds]
    [Prev request in frame: 2642]
    [Prev response in frame: 2644]
    [Request in frame: 2707]
    [Request URI: http://www2b.comm.eng.osaka-u.ac.jp/favicon.ico]
    File Data: 290 bytes
Line-based text data: text/html (9 lines)
    <!DOCTYPE HTML PUBLIC "-//IETF//DTD HTML 2.0//EN">\n
    <html><head>\n
    <title>404 Not Found</title>\n
    </head><body>\n
    <h1>Not Found</h1>\n
    <p>The requested URL was not found on this server.</p>\n
    <hr>\n
    <address>Apache/2.4.41 (Ubuntu) Server at www2b.comm.eng.osaka-u.ac.jp Port 80</address>\n
    </body></html>\n
\end{Verbatim}

 このログから読み取ることができる各パケットの情報を表11と表12にまとめる。
\begin{table}[h]
\centering
\caption{HTTPメッセージの送受信の手続きの分析1}
\begin{tabular}{|c|c|c|c|c|c|}
\hline
No. & 送信元IPアドレス & 受信先IPアドレス & 送信元ポート番号 & 受信先ポート番号 & TCPセグメント長 \\
\hline
2642 & 192.168.12.66 & 192.168.13.80 & 35516 & 80 & 370 \\
2643 & 192.168.13.80 & 192.168.12.66 & 80 & 35516 & 0 \\
2644 & 192.168.13.80 & 192.168.12.66 & 80 & 35516 & 561 \\
2645 & 192.168.12.66 & 192.168.13.80 & 35516 & 80 & 0 \\
2707 & 192.168.12.66 & 192.168.13.80 & 35516 & 80 & 268 \\
2708 & 192.168.13.80 & 192.168.12.66 & 80 & 35516 & 0 \\
2709 & 192.168.13.80 & 192.168.12.66 & 80 & 35516 & 506 \\
2710 & 192.168.12.66 & 192.168.13.80 & 35516 & 80 & 0 \\
\hline
\end{tabular}
\end{table}

\begin{table}[h]
\centering
\caption{HTTPメッセージの送受信の手続きの分析2}
\begin{tabular}{|c|c|c|c|}
\hline
No. & ONになっているTCPフラグ & シーケンス番号 & ACK番号 \\
\hline
2642 & PSH, ACK & 3906406003 & 35308252 \\
2643 & ACK & 35308252 & 3906406373 \\
2644 & PSH, ACK & 35308252 & 3906406373 \\
2645 & ACK & 3906406373 & 35308813 \\
2707 & PSH, ACK & 3906406373 & 35308813 \\
2708 & ACK & 35308813 & 3906406641 \\
2709 & PSH, ACK & 35308813 & 3906406641 \\
2710 & ACK & 3906406641 & 35309319 \\
\hline
\end{tabular}
\end{table}

アプリケーション層のメッセージの内容を以下に示す。
 2642: Webサーバに「HTTP GETメッセージ」を送信して/~yoshiaki/C1/page1.htmlのページ を要求。gzip形式またはdeflate形式での圧縮要求。
 2643 : なし
 2644 : 「HTTP GET メッセージ」に対する「HTTP Response メッセージ」を送信。
   http://www2b.comm.eng.osaka-u.ac.jpのソースコードを含む。gzip形式で圧縮されている。
 2645 : なし
 2707: Webサーバに「HTTP GETメッセージ」を送信してwww2b.comm.eng.osaka-u.ac.jpのfavicon.icoを要求している。
 2708 : なし
 2709 : 「HTTP GET メッセージ」に対する「HTTP Response メッセージ」を送信。
    404 Not Found
 2710 : なし
 No.2642でHTTP GETにより、htmlを圧縮形式で要求し、No.2644でその応用として圧縮されたhtmlが返された。No.2707ではfavicon.icoの取得要求が送られたが、No.2709で「404 Not Found」が返された。なお、No.2642~2645は実験課題2のNo.8354~8357と同様の役割を持つ。
favicon.icoの要求はFirefoxによる自動送信であり、2.4.3では見られなかった。

\subsubsection{TCPコネクションの終了}
 ログでこの手続きを行っているのは、以下の部分である。
\begin{Verbatim}[breaklines=true]
  3222 40.534947   192.168.13.80         192.168.12.66         TCP      66     80 → 35516 [FIN, ACK] Seq=35309319 Ack=3906406641 Win=64640 Len=0 TSval=1939602374 TSecr=3543357075
  3223 40.535294   192.168.12.66         192.168.13.80         TCP      66     35516 → 80 [FIN, ACK] Seq=3906406641 Ack=35309320 Win=64128 Len=0 TSval=3543362075 TSecr=1939602374
  3224 40.535752   192.168.13.80         192.168.12.66         TCP      66     80 → 35516 [ACK] Seq=35309320 Ack=3906406642 Win=64640 Len=0 TSval=1939602374 TSecr=3543362075
\end{Verbatim}

 このログから読み取ることができる各パケットの情報を表13と表14にまとめる。
\begin{table}[h]
\centering
\caption{TCPコネクションを終了する手続きの分析1}
\begin{tabular}{|c|c|c|c|c|c|}
\hline
No. & 送信元IPアドレス & 受信先IPアドレス & 送信元ポート番号 & 受信先ポート番号 & TCPセグメント長 \\
\hline
3222 & 192.168.13.80 & 192.168.12.66 & 80 & 35516 & 0 \\
3223 & 192.168.12.66 & 192.168.13.80 & 35516 & 80 & 0 \\
3224 & 192.168.13.80 & 192.168.12.66 & 80 & 35516 & 0 \\
\hline
\end{tabular}
\end{table}

\begin{table}[h]
\centering
\caption{TCPコネクションを終了する手続きの分析2}
\begin{tabular}{|c|c|c|c|}
\hline
No. & ONになっているTCPフラグ & シーケンス番号 & ACK番号 \\
\hline
3222 & ACK, FIN & 35309319 & 3906406641 \\
3223 & ACK, FIN & 3906406641 & 35309320 \\
3224 & ACK & 35309320 & 3906406642 \\
\hline
\end{tabular}
\end{table}

 全体的な流れは実験課題2のNo.8358〜No.8360と似ているが、今回はWebサーバが先にFINを送信している点が大きな違いである。実験課題2では、クライアントが通信終了を判断したのに対し、実験課題3ではfavicon.icoの応答後にWebサーバが終了を判断しているためである。なお、この3つのパケットにはいずれもアプリケーション層のメッセージは存在しない。

\subsection{実験課題4}
 この課題では、実験課題2、実験課題3と同様の手順でWebブラウザ(Firefox)から以下にアクセスした際の通信ログを分析する。
\url{http://www2b.comm.eng.osaka-u.ac.jp/~yoshiaki/C1/page2.html}  

取得したのは以下のログである。このログを分析し、結果をまとめる。
\begin{Verbatim}[breaklines=true]
No.     Time        Source                Destination           Protocol Length Info
   1019 18.650095   192.168.12.66         10.144.0.1            DNS      88     Standard query 0xfa8f A www2b.comm.eng.osaka-u.ac.jp
   1020 18.651080   10.144.0.1            192.168.12.66         DNS      104    Standard query response 0xfa8f A www2b.comm.eng.osaka-u.ac.jp A 192.168.13.80
   1032 18.760729   192.168.12.66         192.168.13.80         TCP      74     35530 → 80 [SYN] Seq=268385703 Win=64240 Len=0 MSS=1460 \texttt{SACK\_PERM}=1 TSval=3543596857 TSecr=0 WS=128
   1033 18.761271   192.168.13.80         192.168.12.66         TCP      74     80 → 35530 [SYN, ACK] Seq=4149200285 Ack=268385704 Win=65160 Len=0 MSS=1460 \texttt{SACK\_PERM}=1 TSval=1939837156 TSecr=3543596857 WS=128
   1034 18.761396   192.168.12.66         192.168.13.80         TCP      66     35530 → 80 [ACK] Seq=268385704 Ack=4149200286 Win=64256 Len=0 TSval=3543596857 TSecr=1939837156
   1035 18.761708   192.168.12.66         192.168.13.80         HTTP     436    GET /~yoshiaki/C1/page2.html HTTP/1.1 
   1036 18.762236   192.168.13.80         192.168.12.66         TCP      66     80 → 35530 [ACK] Seq=4149200286 Ack=268386074 Win=64896 Len=0 TSval=1939837157 TSecr=3543596858
   1037 18.763016   192.168.13.80         192.168.12.66         HTTP     635    HTTP/1.1 200 OK  (text/html)
   1038 18.763111   192.168.12.66         192.168.13.80         TCP      66     35530 → 80 [ACK] Seq=268386074 Ack=4149200855 Win=64000 Len=0 TSval=3543596859 TSecr=1939837158
   1084 19.170575   192.168.12.66         192.168.13.80         HTTP     420    GET /~yoshiaki/C1/helloworld.png HTTP/1.1 
   1085 19.171240   192.168.13.80         192.168.12.66         TCP      66     80 → 35530 [ACK] Seq=4149200855 Ack=268386428 Win=64640 Len=0 TSval=1939837566 TSecr=3543597267
   1086 19.171658   192.168.13.80         192.168.12.66         TCP      1514   80 → 35530 [ACK] Seq=4149200855 Ack=268386428 Win=64640 Len=1448 TSval=1939837566 TSecr=3543597267 [TCP segment of a reassembled PDU]
   1087 19.171759   192.168.12.66         192.168.13.80         TCP      66     35530 → 80 [ACK] Seq=268386428 Ack=4149202303 Win=64000 Len=0 TSval=3543597268 TSecr=1939837566
   1088 19.171891   192.168.13.80         192.168.12.66         HTTP     659    HTTP/1.1 200 OK  (PNG)
   1089 19.171954   192.168.12.66         192.168.13.80         TCP      66     35530 → 80 [ACK] Seq=268386428 Ack=4149202896 Win=64000 Len=0 TSval=3543597268 TSecr=1939837566
   1116 19.299075   192.168.12.66         192.168.13.80         HTTP     334    GET /favicon.ico HTTP/1.1 
   1117 19.299679   192.168.13.80         192.168.12.66         TCP      66     80 → 35530 [ACK] Seq=4149202896 Ack=268386696 Win=64384 Len=0 TSval=1939837694 TSecr=3543597395
   1118 19.300124   192.168.13.80         192.168.12.66         HTTP     572    HTTP/1.1 404 Not Found  (text/html)
   1119 19.300232   192.168.12.66         192.168.13.80         TCP      66     35530 → 80 [ACK] Seq=268386696 Ack=4149203402 Win=64000 Len=0 TSval=3543597396 TSecr=1939837695
   1643 24.300291   192.168.13.80         192.168.12.66         TCP      66     80 → 35530 [FIN, ACK] Seq=4149203402 Ack=268386696 Win=64384 Len=0 TSval=1939842695 TSecr=3543597396
   1644 24.300712   192.168.12.66         192.168.13.80         TCP      66     35530 → 80 [FIN, ACK] Seq=268386696 Ack=4149203403 Win=64128 Len=0 TSval=3543602397 TSecr=1939842695
   1645 24.301194   192.168.13.80         192.168.12.66         TCP      66     80 → 35530 [ACK] Seq=4149203403 Ack=268386697 Win=64384 Len=0 TSval=1939842696 TSecr=3543602397
\end{Verbatim}

\subsubsection{ドメイン名をIPアドレスに変換}
ログでこの手続きを行っているのは、以下の部分である。
\begin{Verbatim}[breaklines=true]
  1019 18.650095   192.168.12.66         10.144.0.1            DNS      88     Standard query 0xfa8f A www2b.comm.eng.osaka-u.ac.jp
  1020 18.651080   10.144.0.1            192.168.12.66         DNS      104    Standard query response 0xfa8f A www2b.comm.eng.osaka-u.ac.jp A 192.168.13.80
\end{Verbatim}

 次に、パケットキャプチャログの詳細を抜粋して示す。
・1019
\begin{Verbatim}[breaklines=true]
User Datagram Protocol, Src Port: 35759, Dst Port: 53
    Source Port: 35759
    Destination Port: 53
    Length: 54
    Checksum: 0xd7c2 [unverified]
    [Checksum Status: Unverified]
    [Stream index: 15]
Domain Name System (query)
    Transaction ID: 0xfa8f
    Flags: 0x0100 Standard query
(中略)
    Queries
        www2b.comm.eng.osaka-u.ac.jp: type A, class IN
            Name: www2b.comm.eng.osaka-u.ac.jp
            [Name Length: 28]
            [Label Count: 6]
            Type: A (Host Address) (1)
            Class: IN (0x0001)
    [Response In: 1020]
\end{Verbatim}

・1020
\begin{Verbatim}[breaklines=true]
User Datagram Protocol, Src Port: 53, Dst Port: 35759
    Source Port: 53
    Destination Port: 35759
    Length: 70
    Checksum: 0xb66a [unverified]
    [Checksum Status: Unverified]
    [Stream index: 15]
Domain Name System (response)
    Transaction ID: 0xfa8f
    Flags: 0x8580 Standard query response, No error
(中略)
    Queries
        www2b.comm.eng.osaka-u.ac.jp: type A, class IN
            Name: www2b.comm.eng.osaka-u.ac.jp
            [Name Length: 28]
            [Label Count: 6]
            Type: A (Host Address) (1)
            Class: IN (0x0001)
    Answers
        www2b.comm.eng.osaka-u.ac.jp: type A, class IN, addr 192.168.13.80
            Name: www2b.comm.eng.osaka-u.ac.jp
            Type: A (Host Address) (1)
            Class: IN (0x0001)
            Time to live: 1 (1 second)
            Data length: 4
            Address: 192.168.13.80
    [Request In: 1019]
    [Time: 0.000984841 seconds]
\end{Verbatim}

 分析結果を表15にまとめる。
\begin{table}[h]
\centering
\caption{ドメイン名をIPアドレスに変換する手続きの分析}
\begin{tabular}{|c|c|c|c|c|c|}
\hline
No. & 送信元IPアドレス & 受信先IPアドレス & 送信元ポート番号 & 受信先ポート番号 & UDPセグメント長 \\
\hline
1019 & 192.168.12.66 & 10.144.0.1 & 35759 & 53 & 54 \\
1020 & 10.144.0.1 & 192.168.12.66 & 53 & 35759 & 70 \\
\hline
\end{tabular}
\end{table}
アプリケーション層のメッセージの内容を以下に示す。
 1019 : www2b.comm.eng.osaka-u.ac.jpのWebページを保持するIPアドレスの問い合わせ
 1020 : www2b.comm.eng.osaka-u.ac.jpのWebページを保持するIPアドレスを送信
 以上より、この2つのパケットの通信によって、クライアントは入力されたWebサーバに対応するIPアドレス(192.168.13.80)を取得することができることがわかる。実験課題2、実験課題3と同じ流れであることが確認できた。

\subsubsection{TCPコネクションの確立}
 ログでこの手続きを行っているのは、以下の部分である。
\begin{Verbatim}[breaklines=true]
  1032 18.760729   192.168.12.66         192.168.13.80         TCP      74     35530 → 80 [SYN] Seq=268385703 Win=64240 Len=0 MSS=1460 \texttt{SACK\_PERM}=1 TSval=3543596857 TSecr=0 WS=128
  1033 18.761271   192.168.13.80         192.168.12.66         TCP      74     80 → 35530 [SYN, ACK] Seq=4149200285 Ack=268385704 Win=65160 Len=0 MSS=1460 \texttt{SACK\_PERM}=1 TSval=1939837156 TSecr=3543596857 WS=128
  1034 18.761396   192.168.12.66         192.168.13.80         TCP      66     35530 → 80 [ACK] Seq=268385704 Ack=4149200286 Win=64256 Len=0 TSval=3543596857 TSecr=1939837156
\end{Verbatim}

次に、No.1032のパケットキャプチャログの詳細を抜粋して示す。
\begin{Verbatim}[breaklines=true]
Transmission Control Protocol, Src Port: 35530, Dst Port: 80, Seq: 268385703, Len: 0
(中略)
    Acknowledgment number: 0
\end{Verbatim}

ログから読み取ることができる情報を表16、表17にまとめる。
\begin{table}[h]
\centering
\caption{TCPコネクションを確立する手続きの分析1}
\begin{tabular}{|c|c|c|c|c|c|}
\hline
No. & 送信元IPアドレス & 受信先IPアドレス & 送信元ポート番号 & 受信先ポート番号 & TCPセグメント長 \\
\hline
1032 & 192.168.12.66 & 192.168.13.80 & 35530 & 80 & 0 \\
1033 & 192.168.13.80 & 192.168.12.66 & 80 & 35530 & 0 \\
1034 & 192.168.12.66 & 192.168.13.80 & 35530 & 80 & 0 \\
\hline
\end{tabular}
\end{table}

\begin{table}[h]
\centering
\caption{TCPコネクションを確立する手続きの分析2}
\begin{tabular}{|c|c|c|c|}
\hline
No. & ONになっているTCPフラグ & シーケンス番号 & ACK番号 \\
\hline
1032 & SYN & 268385703 & 0 \\
1033 & SYN, ACK & 4149200285 & 268385704 \\
1034 & ACK & 268385704 & 4149200286 \\
\hline
\end{tabular}
\end{table}

 No.1032のパケットでは、クライアントがWebサーバにSYNを送信、No.1032のSYNに対する確認応答(ACK)とSYNが送信されていることがわかる。同様に、No.1033のSYNに対するACKが送信されていることがわかる。なお、この3つのパケットにはアプリケーション層のメッセージが存在しない。実験課題2、実験課題3と同じ手順であることが確認できる。

\subsubsection{HTTPメッセージの送受信}
 ログでこの手続きを行っているのは、以下の部分である。
\begin{Verbatim}[breaklines=true]
  1035 18.761708   192.168.12.66         192.168.13.80         HTTP     436    GET /~yoshiaki/C1/page2.html HTTP/1.1 
   1036 18.762236   192.168.13.80         192.168.12.66         TCP      66     80 → 35530 [ACK] Seq=4149200286 Ack=268386074 Win=64896 Len=0 TSval=1939837157 TSecr=3543596858
   1037 18.763016   192.168.13.80         192.168.12.66         HTTP     635    HTTP/1.1 200 OK  (text/html)
   1038 18.763111   192.168.12.66         192.168.13.80         TCP      66     35530 → 80 [ACK] Seq=268386074 Ack=4149200855 Win=64000 Len=0 TSval=3543596859 TSecr=1939837158
   1084 19.170575   192.168.12.66         192.168.13.80         HTTP     420    GET /~yoshiaki/C1/helloworld.png HTTP/1.1 
   1085 19.171240   192.168.13.80         192.168.12.66         TCP      66     80 → 35530 [ACK] Seq=4149200855 Ack=268386428 Win=64640 Len=0 TSval=1939837566 TSecr=3543597267
   1086 19.171658   192.168.13.80         192.168.12.66         TCP      1514   80 → 35530 [ACK] Seq=4149200855 Ack=268386428 Win=64640 Len=1448 TSval=1939837566 TSecr=3543597267 [TCP segment of a reassembled PDU]
   1087 19.171759   192.168.12.66         192.168.13.80         TCP      66     35530 → 80 [ACK] Seq=268386428 Ack=4149202303 Win=64000 Len=0 TSval=3543597268 TSecr=1939837566
   1088 19.171891   192.168.13.80         192.168.12.66         HTTP     659    HTTP/1.1 200 OK  (PNG)
   1089 19.171954   192.168.12.66         192.168.13.80         TCP      66     35530 → 80 [ACK] Seq=268386428 Ack=4149202896 Win=64000 Len=0 TSval=3543597268 TSecr=1939837566
   1116 19.299075   192.168.12.66         192.168.13.80         HTTP     334    GET /favicon.ico HTTP/1.1 
   1117 19.299679   192.168.13.80         192.168.12.66         TCP      66     80 → 35530 [ACK] Seq=4149202896 Ack=268386696 Win=64384 Len=0 TSval=1939837694 TSecr=3543597395
   1118 19.300124   192.168.13.80         192.168.12.66         HTTP     572    HTTP/1.1 404 Not Found  (text/html)
   1119 19.300232   192.168.12.66         192.168.13.80         TCP      66     35530 → 80 [ACK] Seq=268386696 Ack=4149203402 Win=64000 Len=0 TSval=3543597396 TSecr=1939837695
\end{Verbatim}

 次に、それぞれのパケットキャプチャログの詳細を抜粋して示す。
・1035
\begin{Verbatim}[breaklines=true]
Transmission Control Protocol, Src Port: 35530, Dst Port: 80, Seq: 268385704, Ack: 4149200286, Len: 370
(中略)
    Flags: 0x018 (PSH, ACK)
(中略)
Hypertext Transfer Protocol
    GET /~yoshiaki/C1/page2.html HTTP/1.1\textbackslash r\textbackslash n
(中略)
    Host: www2b.comm.eng.osaka-u.ac.jp\textbackslash r\textbackslash n
    User-Agent: Mozilla/5.0 (X11; Linux armv7l; rv:78.0) Gecko/20100101 Firefox/78.0\textbackslash r\textbackslash n
    Accept: text/html,application/xhtml+xml,application/xml;q=0.9,image/webp,*/*;q=0.8\textbackslash r\textbackslash n
    Accept-Language: en-US,en;q=0.5\textbackslash r\textbackslash n
    Accept-Encoding: gzip, deflate\textbackslash r\textbackslash n
    DNT: 1\textbackslash r\textbackslash n
    Connection: keep-alive\textbackslash r\textbackslash n
    Upgrade-Insecure-Requests: 1\textbackslash r\textbackslash n
    \textbackslash r\textbackslash n
    [Full request URI: http://www2b.comm.eng.osaka-u.ac.jp/~yoshiaki/C1/page2.html]
    [HTTP request 1/3]
    [Response in frame: 1037]
    [Next request in frame: 1084]
\end{Verbatim}

・1037
\begin{Verbatim}[breaklines=true]
Transmission Control Protocol, Src Port: 80, Dst Port: 35530, Seq: 4149200286, Ack: 268386074, Len: 569
(中略)
    Flags: 0x018 (PSH, ACK)
(中略)
Hypertext Transfer Protocol
    HTTP/1.1 200 OK\textbackslash r\textbackslash n
(中略)
    Date: Tue, 11 May 2021 05:59:53 GMT\textbackslash r\textbackslash n
    Server: Apache/2.4.41 (Ubuntu)\textbackslash r\textbackslash n
    Last-Modified: Thu, 17 Sep 2020 03:12:39 GMT\textbackslash r\textbackslash n
    ETag: "119-5af79c1bbefeb-gzip"\textbackslash r\textbackslash n
    Accept-Ranges: bytes\textbackslash r\textbackslash n
    Vary: Accept-Encoding\textbackslash r\textbackslash n
    Content-Encoding: gzip\textbackslash r\textbackslash n
    Content-Length: 232\textbackslash r\textbackslash n
        [Content length: 232]
    Keep-Alive: timeout=5, max=100\textbackslash r\textbackslash n
    Connection: Keep-Alive\textbackslash r\textbackslash n
    Content-Type: text/html\textbackslash r\textbackslash n
    \textbackslash r\textbackslash n
\end{Verbatim}

・1084
\begin{Verbatim}[breaklines=true]
Transmission Control Protocol, Src Port: 35530, Dst Port: 80, Seq: 268386074, Ack: 4149200855, Len: 354 
(中略)
    Flags: 0x018 (PSH, ACK) 
(中略)
Hypertext Transfer Protocol
    GET /~yoshiaki/C1/helloworld.png HTTP/1.1\textbackslash r\textbackslash n
(中略)
    Host: www2b.comm.eng.osaka-u.ac.jp\textbackslash r\textbackslash n
    User-Agent: Mozilla/5.0 (X11; Linux armv7l; rv:78.0) Gecko/20100101 Firefox/78.0\textbackslash r\textbackslash n
    Accept: image/webp,*/*\textbackslash r\textbackslash n
    Accept-Language: en-US,en;q=0.5\textbackslash r\textbackslash n
    Accept-Encoding: gzip, deflate\textbackslash r\textbackslash n
    DNT: 1\textbackslash r\textbackslash n
    Connection: keep-alive\textbackslash r\textbackslash n
    Referer: http://www2b.comm.eng.osaka-u.ac.jp/~yoshiaki/C1/page2.html\textbackslash r\textbackslash n
    \textbackslash r\textbackslash n
    [Full request URI: http://www2b.comm.eng.osaka-u.ac.jp/~yoshiaki/C1/helloworld.png]
    [HTTP request 2/3]
    [Prev request in frame: 1035]
    [Response in frame: 1088]
    [Next request in frame: 1116]
\end{Verbatim}

・1088
\begin{Verbatim}[breaklines=true]
Transmission Control Protocol, Src Port: 80, Dst Port: 35530, Seq: 4149202303, Ack: 268386428, Len: 593 
(中略)
    Flags: 0x018 (PSH, ACK)
(中略)
Hypertext Transfer Protocol
    HTTP/1.1 200 OK\textbackslash r\textbackslash n 
(中略)
    Date: Tue, 11 May 2021 05:59:54 GMT\textbackslash r\textbackslash n
    Server: Apache/2.4.41 (Ubuntu)\textbackslash r\textbackslash n
    Last-Modified: Thu, 17 Sep 2020 03:12:39 GMT\textbackslash r\textbackslash n
    ETag: "6dc-5af79c1ba3a6b"\textbackslash r\textbackslash n
    Accept-Ranges: bytes\textbackslash r\textbackslash n
    Content-Length: 1756\textbackslash r\textbackslash n
        [Content length: 1756]
    Keep-Alive: timeout=5, max=99\textbackslash r\textbackslash n
    Connection: Keep-Alive\textbackslash r\textbackslash n
    Content-Type: image/png\textbackslash r\textbackslash n
    \textbackslash r\textbackslash n
\end{Verbatim}

・1116
\begin{Verbatim}[breaklines=true]
Transmission Control Protocol, Src Port: 35530, Dst Port: 80, Seq: 268386428, Ack: 4149202896, Len: 268 
(中略)
    Flags: 0x018 (PSH, ACK)
(中略)
Hypertext Transfer Protocol
    GET /favicon.ico HTTP/1.1\textbackslash r\textbackslash n
(中略)
    Host: www2b.comm.eng.osaka-u.ac.jp\textbackslash r\textbackslash n
    User-Agent: Mozilla/5.0 (X11; Linux armv7l; rv:78.0) Gecko/20100101 Firefox/78.0\textbackslash r\textbackslash n
    Accept: image/webp,*/*\textbackslash r\textbackslash n
    Accept-Language: en-US,en;q=0.5\textbackslash r\textbackslash n
    Accept-Encoding: gzip, deflate\textbackslash r\textbackslash n
    DNT: 1\textbackslash r\textbackslash n
    Connection: keep-alive\textbackslash r\textbackslash n
    \textbackslash r\textbackslash n
    [Full request URI: http://www2b.comm.eng.osaka-u.ac.jp/favicon.ico]
    [HTTP request 3/3]
    [Prev request in frame: 1084]
    [Response in frame: 1118]
\end{Verbatim}

・1118
\begin{Verbatim}[breaklines=true]
Transmission Control Protocol, Src Port: 80, Dst Port: 35530, Seq: 4149202896, Ack: 268386696, Len: 506 
(中略)
    Flags: 0x018 (PSH, ACK)
(中略)
Hypertext Transfer Protocol
    HTTP/1.1 404 Not Found\textbackslash r\textbackslash n
        [Expert Info (Chat/Sequence): HTTP/1.1 404 Not Found\textbackslash r\textbackslash n]
            [HTTP/1.1 404 Not Found\textbackslash r\textbackslash n]
            [Severity level: Chat]
            [Group: Sequence]
        Response Version: HTTP/1.1
        Status Code: 404
        [Status Code Description: Not Found]
        Response Phrase: Not Found
    Date: Tue, 11 May 2021 05:59:54 GMT\textbackslash r\textbackslash n
    Server: Apache/2.4.41 (Ubuntu)\textbackslash r\textbackslash n
    Content-Length: 290\textbackslash r\textbackslash n
        [Content length: 290]
    Keep-Alive: timeout=5, max=98\textbackslash r\textbackslash n
    Connection: Keep-Alive\textbackslash r\textbackslash n
    Content-Type: text/html; charset=iso-8859-1\textbackslash r\textbackslash n
\end{Verbatim}

このログから読み取ることができる各パケットの情報を表18と表19にまとめる。

\begin{table}[h]
\centering
\caption{HTTPメッセージの送受信の手続きの分析1}
\begin{tabular}{|c|c|c|c|c|c|}
\hline
No. & 送信元IPアドレス & 受信先IPアドレス & 送信元ポート番号 & 受信先ポート番号 & TCPセグメント長 \\
\hline
1035 & 192.168.12.66 & 192.168.13.80 & 35530 & 80 & 370 \\
1036 & 192.168.13.80 & 192.168.12.66 & 80 & 35530 & 0 \\
1037 & 192.168.13.80 & 192.168.12.66 & 80 & 35530 & 569 \\
1038 & 192.168.12.66 & 192.168.13.80 & 35530 & 80 & 0 \\
1084 & 192.168.12.66 & 192.168.13.80 & 35530 & 80 & 354 \\
1085 & 192.168.13.80 & 192.168.12.66 & 80 & 35530 & 0 \\
1086 & 192.168.13.80 & 192.168.12.66 & 80 & 35530 & 1448 \\
1087 & 192.168.12.66 & 192.168.13.80 & 35530 & 80 & 0 \\
1088 & 192.168.13.80 & 192.168.12.66 & 80 & 35530 & 593 \\
1089 & 192.168.12.66 & 192.168.13.80 & 35530 & 80 & 0 \\
1116 & 192.168.12.66 & 192.168.13.80 & 35530 & 80 & 268 \\
1117 & 192.168.13.80 & 192.168.12.66 & 80 & 35530 & 0 \\
1118 & 192.168.13.80 & 192.168.12.66 & 80 & 35530 & 506 \\
1119 & 192.168.12.66 & 192.168.13.80 & 35530 & 80 & 0 \\
\hline
\end{tabular}
\end{table}

\begin{table}[h]
\centering
\caption{HTTPメッセージの送受信の手続きの分析2}
\begin{tabular}{|c|c|c|c|}
\hline
No. & ONになっているTCPフラグ & シーケンス番号 & ACK番号 \\
\hline
1035 & PSH, ACK & 268385704 & 4149200286 \\
1036 & ACK & 4149200286 & 268386074 \\
1037 & PSH, ACK & 4149200286 & 268386074 \\
1038 & ACK & 268386074 & 4149200855 \\
1084 & PSH, ACK & 268386074 & 4149200855 \\
1085 & ACK & 4149200855 & 268386428 \\
1086 & PSH, ACK & 4149200855 & 268386428 \\
1087 & ACK & 268386428 & 4149202303 \\
1088 & PSH, ACK & 4149202303 & 268386428 \\
1089 & ACK & 268386428 & 4149202896 \\
1116 & PSH, ACK & 268386428 & 4149202896 \\
1117 & ACK & 4149202896 & 268386696 \\
1118 & PSH, ACK & 4149202896 & 268386696 \\
1119 & ACK & 268386696 & 4149203402 \\
\hline
\end{tabular}
\end{table}

アプリケーション層のメッセージの内容を以下に示す。
 1035: Webサーバに「HTTP GETメッセージ」を送信しhtml要求。
 1036 : なし
 1037 : 「HTTP GET メッセージ」に対する「HTTP Response メッセージ」を送信。
   http://www2b.comm.eng.osaka-u.ac.jpのソースコードを含む。
 1038 : なし
 1084: Webサーバに「HTTP GETメッセージ」を送信して/~yoshiaki/C1/helloworld.pngの画像ファイルをgzipまたはdeflateで圧縮することを要求している。
 1085 : なし
 1086 : なし
 1087 : なし
 1088 : 「HTTP GET メッセージ」に対する「HTTP Response メッセージ」を送信。
   /~yoshiaki/C1/helloworld.pngのHTMLファイルを含んでいる。
 1089 : なし
 1116: Webサーバに「HTTP GETメッセージ」を送信してfavicon.icoを要求。
 1117 : なし
 1118 : 「HTTP GET メッセージ」に対する「HTTP Response メッセージ」を送信。
   404 Not Found (favicon.icoが見つからなかった)
 1119 : なし
No.1035~No.1038は実験課題2・3と同様の通信である。No.1084〜No.1089では、画像ファイル” helloworld.png”を要求・取得しており、これは画像を含むページ特有の通信である。No.1116〜No.1119ではfavicon.icoの取得を試みたが失敗しており、実験課題3と同じである。

\subsubsection{TCPコネクションの終了}
 ログでこの手続きを行っているのは、以下の部分である。
\begin{Verbatim}[breaklines=true]
  1643 24.300291   192.168.13.80         192.168.12.66         TCP      66     80 → 35530 [FIN, ACK] Seq=4149203402 Ack=268386696 Win=64384 Len=0 TSval=1939842695 TSecr=3543597396
  1644 24.300712   192.168.12.66         192.168.13.80         TCP      66     35530 → 80 [FIN, ACK] Seq=268386696 Ack=4149203403 Win=64128 Len=0 TSval=3543602397 TSecr=1939842695
  1645 24.301194   192.168.13.80         192.168.12.66         TCP      66     80 → 35530 [ACK] Seq=4149203403 Ack=268386697 Win=64384 Len=0 TSval=1939842696 TSecr=3543602397
\end{Verbatim}

 このログから読み取ることができる各パケットの情報を表20と表21にまとめる。

\begin{table}[h]
\centering
\caption{TCPコネクションを終了する手続きの分析1}
\begin{tabular}{|c|c|c|c|c|c|}
\hline
No. & 送信元IPアドレス & 受信先IPアドレス & 送信元ポート番号 & 受信先ポート番号 & TCPセグメント長 \\
\hline
1643 & 192.168.13.80 & 192.168.12.66 & 80 & 35530 & 0 \\
1644 & 192.168.12.66 & 192.168.13.80 & 35530 & 80 & 0 \\
1645 & 192.168.13.80 & 192.168.12.66 & 80 & 35530 & 0 \\
\hline
\end{tabular}
\end{table}

\begin{table}[h]
\centering
\caption{TCPコネクションを終了する手続きの分析2}
\begin{tabular}{|c|c|c|c|}
\hline
No. & ONになっているTCPフラグ & シーケンス番号 & ACK番号 \\
\hline
1643 & ACK, FIN & 4149203402 & 268386696 \\
1644 & ACK, FIN & 268386696 & 4149203403 \\
1645 & ACK & 4149203403 & 268386697 \\
\hline
\end{tabular}
\end{table}

 No.1643からNo.1645のパケットは、実験課題2、実験課題3と同じ処理を行なっていることがわかる。この3つのパケットにはいずれもアプリケーション層のメッセージは存在しない。



\section{第二週:Webアプリケーションにおける輻輳現象の分析}
 画像識別を行うWebアプリケーションを題材として、情報システムにおいて生じる輻輳現象を観察する。クライアントとWebサーバはLANケーブルとスイッチングハブを介して接続しており、Webサーバ上では画像識別を行うWebアプリケーションが動作している。クライアントからWebサーバへHTTP POSTメッセージとして画像ファイルを送信すると、Webサーバは「ResNet V2 101」と呼ばれる深層ニューラルネットワークを使用して画像のラベルを識別し、識別結果をHTTP Responseメッセージとしてクライアントに返す。今回実験対象とするシステムの概略図は以下の図2である。

\begin{figure}[H]
\centering
\includegraphics[width=0.8\textwidth]{images/図2.png}
\caption{第2週で用いるシステムの模式図}
\end{figure}
 このシステムでは、クライアントが画像を送信可能である最小間隔と比べて、Webサーバが画像識別に要する時間の方がはるかに大きいという特徴がある。したがって、クライアントが最大の送信頻度で画像を送り続けるとWebサーバの処理能力を超過してしまい、遅延時間が極めて増大するという結果を招く。本実験では、このような状況において、送信頻度が遅延時間に与える影響、並びに遅延時間との適切な兼ね合いをとることができる適切な送信頻度の決定方法について、実測に基づいた考察を行う。

\subsection{理論}
 本実験のシステムは、待ち行列モデルを用いてモデル化される。本実験で考える単一サーバ待ち行列モデルは、待合室と単一のサーバで構成され、外部から次々に到着するジョブをサーバが順番に処理する。このモデルの模式図を図3に示す。

\begin{figure}[H]
\centering
\includegraphics[width=0.8\textwidth]{images/図3.png}
\caption{単一サーバ待ち行列モデル}
\end{figure}
 単一サーバ待ち行列モデルは、具体的には以下の手続きに従って動作する。
(a)ある頻度で、外部からジョブが待合室に到着する。
 (a-1)ジョブの到着時点においてサーバが空いていた場合、
到着ジョブは即座にサーバで処理を受け始める。
 (a-2)ジョブの到着時点において他のジョブが処理中であった場合、
到着ジョブは待合室で待機する。
(b)ジョブの処理が完了した時点において、
 (b-1)待合室にジョブが存在する場合、サーバは次のジョブの処理を開始する。
 (b-2)待合室にジョブが存在しない場合、サーバは次のジョブの到着まで待機する。

 本実験においては、待ち行列モデルにおける「待合室」および「サーバ」は、Webサーバ上に存在する「バッファ」および「演算装置(CPU)」に相当する。また、「外部」とはWebサーバの外側のことを表し、実際には到着ジョブはクライアントが生成してサーバに送出したものである。
時刻0に最初のジョブが到着し、その後$N-1$個のジョブが到着するものとする。$n$番目($n=0,1,\cdots ,N-1$)のジョブの到着時刻を$\alpha_n$ ($\alpha_{n-1}<\alpha_n$)とする。ただし、0番目に到着したジョブについては、$\alpha_0=0$とする。$G_n$を$n-1$番目と$n$番目のジョブの到着間隔、すなわち式(1)のように定義する。
\begin{equation}
G_n=\alpha_n-\alpha_{n-1} \quad ( n=1,2,\cdots ,N-1) \tag{1}
\end{equation}

 次に、$n$番目($n=0,1,\cdots$)のジョブがサービスを完了する時刻を$\beta_n$とし、そのジョブの遅延時間を$D_n$と定義する。なお、遅延時間はジョブが到着してからサービスを完了するまでにかかる時間を表すため、式(2)のように定義できる。
\begin{equation}
D_n=\beta_n-\alpha_n \quad ( n=1,2,\cdots ,N-1) \tag{2}
\end{equation}

$W_n$を他のジョブの処理が完了するまでの待合室での待ち時間、$H_n$をサーバにおけるそのジョブの処理時間を表す。この2つを用いて遅延時間を表すと、式(3)のようになる。
\begin{equation}
D_n=W_n+H_n \quad ( n=1,2,\cdots ,N-1) \tag{3}
\end{equation}

先着順処理の単一サーバ待ち行列では、次の漸化式(Lindley方程式)が成立する。
\begin{equation}
D_n=\max(0,D_{n-1} - G_n )+H_n \quad ( n=1,2,\cdots ,N-1) \tag{4}
\end{equation}

式(4)を変形することにより、以下の式(5)が得られる。
\begin{equation}
D_n=\max(G_n,D_{n-1} )+H_n-G_n \quad ( n=1,2,\cdots ,N-1) \tag{5}
\end{equation}

この式において$G_n \geq D_{n-1}$となるとき、$n$番目のジョブは待ち時間なく処理を開始できるため、式(6)が成り立つ。
\begin{equation}
D_n=H_n \tag{6}
\end{equation}

一方、$G_n<D_{n-1}$となるときは$n$番目のジョブは1つ前のジョブの処理が完了するのを待つ必要があるため、式(7)のようになる。
\begin{equation}
D_n=D_{n-1}+H_n-G_n \tag{7}
\end{equation}

式(7)より、$G_n<D_{n-1}$の場合には一つ前のジョブの遅延時間$D_{n-1}$と比較して、遅延時間$D_n$は$H_n-G_n$だけ大きくなることがわかる。
従って、ジョブの処理時間$H_n$と比べてジョブの到着間隔$G_n$が小さい状況が継続した場合、遅延時間$D_n$は際限なく増加し続ける。
$A(t)$ ($t \geq 0$)を時刻$t$までに到着したジョブの総数と定義すると、式(8)が成り立つ。ただし、$I(A)$は、$A$が真であるときに1、偽であるときに0をとる関数(指示関数)を表す。
\begin{equation}
A(t)=\sum_{n=0}^{N-1} I\{\alpha_n \leq t\} \tag{8}
\end{equation}

同様に、$D(t)$ ($t \geq 0$)を時刻$t$までにサービスを完了して離脱したジョブの総数と定義する。
\begin{equation}
D(t)=\sum_{n=0}^{N-1} I\{\beta_n \leq t\} \tag{9}
\end{equation}
時刻$t$において、待合室やサーバに滞在しているジョブの総数$L(t)$は式(10)のように表される。
\begin{equation}
L(t)=A(t)-D(t) \quad (t \geq 0) \tag{10}
\end{equation}
 $T$ ($0 \leq T \leq \beta_N$)を最後のジョブが離脱するまでのある一時刻とし、時間$(0,T]$における平均到着率$\bar{\lambda}(T)$と平均離脱率$\bar{\mu}(T)$をそれぞれ式(11)、(12)で定義する。なお、$\bar{\lambda}(T)$は単位時間当たりの到着数、$\bar{\mu}(T)$は単位時間当たりの離脱数を表す。
\begin{equation}
\bar{\lambda}(T)=\frac{A(T)-1}{T} \tag{11}
\end{equation}
\begin{equation}
\bar{\mu}(T)=\frac{D(T)}{T} \tag{12}
\end{equation}

平均到着率$\bar{\lambda}(T)$は到着間隔$G_n$ ($n=1,2,\cdots ,N-1$)の平均と式(13)のように関係付けられる。以下の議論では、$T \geq \alpha_1$を仮定する。平均間隔$\bar{\tau}(T)$を時間$(0,T]$に発生した到着の平均間隔とすると、式(13)のように定義できる。
\begin{equation}
\bar{\tau}(T)=\frac{1}{A(T)-1} \sum_{n=1}^{A(T)-1} G_n \tag{13}
\end{equation}

 $A(T) \leq N-2$のとき、次の不等式が成り立つ。
\begin{equation}
\sum_{n=1}^{A(T)-1} G_n \leq T < \sum_{n=1}^{A(T)-1} G_n + G_{A(T)} \tag{14}
\end{equation}

また、$A(T) \leq N-2$のとき、次の不等式が成り立つ。
\begin{equation}
\bar{\tau}(T) \leq \frac{1}{\bar{\lambda}(T)} < \bar{\tau}(T)+\frac{G_{A(T)}}{A(T)-1} \tag{15}
\end{equation}

 $A(T)$が十分に大きいとき、式(15)の最右辺における第二項は無視できるほど小さいとみなして良いため、式(16)が得られる。このとき、平均到着率は平均到着間隔の逆数にほぼ等しくなる。
\begin{equation}
\bar{\lambda}(T) \cong \frac{1}{\bar{\tau}(T)} \tag{16}
\end{equation}

一方、平均離脱率は到着間隔$G_n$と処理時間$H_n$の両方に依存する。到着ジョブの負荷に対してサーバの処理能力より十分高いとき、すなわち、処理時間が到着間隔に比べて小さいとき、時刻$T$での滞留ジョブ数$L(T)$は0に近似できる。このとき、$A(T) \cong D(T)$となるため、式(17)が成り立つ。
\begin{equation}
\bar{\mu}(T) \cong \bar{\lambda}(T) \tag{17}
\end{equation}

この式はサーバの処理能力が到達ジョブの負荷と比べて高いときにのみ成り立つ。
ここで、時刻$[0,T]$における平均処理時間を$\bar{B}(T)$とし、以下の式(18)で定義する。ただし、$D(T) \geq 1$を仮定する。
\begin{equation}
\bar{B}(T)=\frac{1}{D(T)} \sum_{n=1}^{D(T)-1} H_n \tag{18}
\end{equation}
到着ジョブの負荷がサーバの処理能力を超えてしまった場合、つまり処理時間が到着間隔に比べて大きいときはサーバが常にジョブを処理している状態となる。そのため時刻$t \in [0,T]$において滞留ジョブ数$L(t)$は1以上の値をとる。ジョブの離脱間隔は処理時間と等しくなるため、式(19)が成り立つ。
\begin{equation}
\bar{\mu}(T) \cong \frac{1}{\bar{B}(T)} \tag{19}
\end{equation}
ただしこの式は、サーバの処理能力が到達ジョブの負荷より低いときにのみ成り立つ。
以上より平均離脱率$\bar{\mu}(T)$は以下のような特徴をもつ。
・サーバの処理能力が到着ジョブの負荷と比べて高い
→平均離脱率$\bar{\mu}(T)$と平均到着率$\bar{\lambda}(T)$はほぼ等しい
・サーバの処理能力が到着ジョブの負荷と比べて低い
→平均離脱率$\bar{\mu}(T)$と平均処理時間$\bar{B}(T)$の逆数はほぼ等しい

式(10)、(11)、(12)より、時刻$T$における滞留ジョブ数$L(T)$は式(20)で与えられる。
\begin{equation}
L(T)=(\bar{\lambda}(T)-\bar{\mu}(T)) \times T \tag{20}
\end{equation}

以上より、滞留ジョブ数$L(T)$について次の結論を得る。
・サーバの処理能力が到着ジョブの負荷と比べて高い
→滞留ジョブ数は比較的小さな値にとどまる。
・サーバの処理能力が到着ジョブの負荷と比べて低い
→滞留ジョブ数は時間経過とともに際限なく増加する。

\subsection{演習課題1}
先着順処理の単一サーバ待ち行列では、次の漸化式(Lindley方程式)が成立する。
\begin{equation}
D_n=\max(0,D_{n-1} - G_n )+H_n \quad ( n=1,2,\cdots ,N-1)
\end{equation}

この定理を証明する。
(i) $n$番目のジョブの到着時点においてサーバが空いていた場合
到着後すぐにサーバで処理されることとなり待ち時間は0なので
\begin{equation*}
W_n=0
\end{equation*}
(ii) $n$番目のジョブの到着時点において他のジョブが処理中であった場合
$n$番目のジョブは$n-1$番目のジョブの処理が終了するまで待合室で待機することになる。よって待機時間は、$n$番目のジョブが到着してから$n-1$番目に到着したジョブの処理が終わるまでの時間と一致するため、以下のように表すことができる。
\begin{equation*}
W_n=\beta_{n-1}-\alpha_n=(\beta_{n-1}-\alpha_{n-1} )-(\alpha_n-\alpha_{n-1} )
\end{equation*}
式(1)、(2)より、$W_n$は到着間隔$G_n$と遅延時間$D_n$を用いて以下のように表すことができる。
\begin{equation*}
W_n=D_{n-1}-G_n
\end{equation*}
また、$W_n>0$より$D_{n-1}-G_n>0$である。

(i),(ii)より、
\begin{equation*}
W_n=\max(0,D_{n-1} - G_n )
\end{equation*}
である。式(3)より、
\begin{equation*}
D_n=\max(0,D_{n-1} - G_n )+H_n \quad ( n=1,2,\cdots ,N-1)
\end{equation*}
 
\subsection{演習課題2-補題1の証明}
$A(T) \leq N-2$のとき、次の不等式が成り立つ。
\begin{equation}
\sum_{n=1}^{A(T)-1} G_n \leq T < \sum_{n=1}^{A(T)-1} G_n + G_{A(T)}
\end{equation}

この補題を証明する。
まず、最左辺について考える。式(1)で表した$G_n$の定義より、
\begin{equation*}
\sum_{n=1}^{A(T)-1} G_n = \sum_{n=1}^{A(T)-1} (\alpha_n-\alpha_{n-1}) = (\alpha_1-\alpha_0 )+(\alpha_2-\alpha_1 )+\cdots+(\alpha_{A(T)-1}-\alpha_{A(T)-2})
\end{equation*}

定義より、$\alpha_0=0$なので、
\begin{equation*}
\sum_{n=1}^{A(T)-1} G_n = \alpha_{A(T)-1}-\alpha_0 = \alpha_{A(T)-1}
\end{equation*}

不等式の最右辺についても同様に考える。$G_{A(T)} = \alpha_{A(T)} - \alpha_{A(T)-1}$より
\begin{equation*}
\sum_{n=1}^{A(T)-1} G_n + G_{A(T)} = \alpha_{A(T)-1}+(\alpha_{A(T)} - \alpha_{A(T)-1}) = \alpha_{A(T)}
\end{equation*}

以上より、補題を示すためには以下の不等式(22)が成り立つことを示せばよい。
\begin{equation}
\alpha_{A(T)-1} \leq T < \alpha_{A(T)} \tag{22}
\end{equation}
 $A(T)$は時間$T$までに到着したジョブの総数であるため、0番目から$(A(T)-1)$番目のジョブは時間$T$よりはやく到着していることになる。従って、$(A(T)-1)$番目のジョブの到着時間$\alpha_{A(T)-1}$は$T$以下になる。しかし、$A(T)$番目以降のジョブは時間$T$より遅く到着する。そのため、$\alpha_{A(T)}$は$T$よりも大きくなる。
以上より、不等式(22)は成り立つ。よって、補題1の不等式も成立する。


\subsection{演習課題2-定理2の証明}
$A(T) \leq N-2$のとき、次の不等式が成り立つ。
\begin{equation}
\bar{\tau}(T) \leq \frac{1}{\bar{\lambda}(T)} < \bar{\tau}(T)+\frac{G_{A(T)}}{A(T)-1}
\end{equation}
この定理を証明する。
補題1において示した不等式の各辺を$A(T)-1$で割ると
\begin{equation*}
\frac{1}{A(T)-1} \sum_{n=1}^{A(T)-1} G_n \leq \frac{T}{A(T)-1} < \frac{1}{A(T)-1} \sum_{n=1}^{A(T)-1} G_n + \frac{G_{A(T)}}{A(T)-1}
\end{equation*}

式(11)で表した平均到着率$\bar{\lambda}(T)$の定義と、式(13)で表した時間$(0,T]$に発生した到着の平均間隔$\bar{\tau}(T)$の定義より
\begin{equation*}
\bar{\tau}(T) \leq \frac{1}{\bar{\lambda}(T)} < \bar{\tau}(T)+\frac{G_{A(T)}}{A(T)-1}
\end{equation*}
となる。定理2の不等式も示された。

\subsection{実験環境}
クライアント側ハードウェア
・Raspberry Pi 3 Model B+
クライアント側ソフトウェア
・Ubuntu 20.04 LTS
・Python 3.8.2
サーバ側ハードウェア
・HP ProDesk 400 G4 DM/CT
サーバ側ソフトウェア
・Ubuntu 20.04 LTS
・TensorFlow 2.3.1
・TFServe 0.3, ResNet V
・Python 3.8.2

\subsection{実験課題5}
 この課題では、クライアントでターミナルを開き、ホームディレクトリ直下のclientディレクトリに移動する。その後、以下のコマンドを実行した。
\begin{Verbatim}[breaklines=true]
$ python3 client.py
\end{Verbatim}
 実行結果は以下のようになった。
\begin{Verbatim}[breaklines=true]
ubuntu@raspberrypi3:~/client $ python3 client.py 
converted_500/test_324.png
[{'class': 'German shepherd, German shepherd dog, German police dog, alsatian', 'prob': 21.546550750732422}, {'class': 'kelpie', 'prob': 17.637371063232422}, {'class': 'toy terrier', 'prob': 14.025994300842285}, {'class': 'malinois', 'prob': 11.318921089172363}]
\end{Verbatim}
 実際に送信されたサーバへ送信された画像は\texttt{test\_324.png}である。この画像を図4に示す。

\begin{figure}[H]
\centering
\includegraphics[width=0.8\textwidth]{images/図4.png}
\caption{サーバに送信された画像(\texttt{test\_324.png})}
\end{figure}
 実行結果より、送信された画像はゲルマンシェパードである可能性が最も高いとわかる。\texttt{test\_324.png}も犬の画像であることから識別結果は正しいと考えることができる。

\subsection{実験課題6}
 この課題では、実験課題2と同様の手順で、実験課題5のコマンドを入力したときに行われた一連の通信に関するログを取得する。手順は以下に示す。

1. クライアントでWiresharkを起動し、パケットキャプチャを開始。
2. 実験課題5と同じコマンドを実行。
3. Webページのソースコードが標準出力に表示されたら、パケットキャプチャを停止。

実行結果は以下のようになった。
\begin{Verbatim}[breaklines=true]
No.     Time        Source                Destination           Protocol Length Info
   2140 18.969328   192.168.12.66         10.144.0.1            DNS      92     Standard query 0x3ba2 A exp1.rnea.comm.eng.osaka-u.ac.jp
   2141 18.969419   192.168.12.66         10.144.0.1            DNS      92     Standard query 0x43a2 AAAA exp1.rnea.comm.eng.osaka-u.ac.jp
   2142 18.970377   10.144.0.1            192.168.12.66         DNS      108    Standard query response 0x3ba2 A exp1.rnea.comm.eng.osaka-u.ac.jp A 192.168.12.23
   2143 18.971896   10.144.0.1            192.168.12.66         DNS      144    Standard query response 0x43a2 No such name AAAA exp1.rnea.comm.eng.osaka-u.ac.jp SOA gene.comm.eng.osaka-u.ac.jp
   2144 18.972480   192.168.12.66         192.168.12.23         TCP      74     42522 → 5000 [SYN] Seq=4044280001 Win=64240 Len=0 MSS=1460 SACK_PERM=1 TSval=3712487604 TSecr=0 WS=128
   2145 18.974209   192.168.12.23         192.168.12.66         TCP      74     5000 → 42522 [SYN, ACK] Seq=1438445294 Ack=4044280002 Win=65160 Len=0 MSS=1460 SACK_PERM=1 TSval=2689562944 TSecr=3712487604 WS=128
   2146 18.974294   192.168.12.66         192.168.12.23         TCP      66     42522 → 5000 [ACK] Seq=4044280002 Ack=1438445295 Win=64256 Len=0 TSval=3712487606 TSecr=2689562944
   2147 18.974476   192.168.12.66         192.168.12.23         IPA      227    unknown 0x53 
   2148 18.974618   192.168.12.66         192.168.12.23         IPA      1514   unknown 0x4e 
   2149 18.974640   192.168.12.66         192.168.12.23         IPA      1514   unknown 0x6e 
   2150 18.974654   192.168.12.66         192.168.12.23         IPA      1514   unknown 0xbc 
   2151 18.974667   192.168.12.66         192.168.12.23         IPA      1514   unknown 0x23 
   2152 18.974681   192.168.12.66         192.168.12.23         IPA      1514   unknown 0xb5 
   2153 18.974734   192.168.12.66         192.168.12.23         RSL      1514   ip.access PDCH DEACTIVATION 
   2154 18.974761   192.168.12.66         192.168.12.23         IPA      1514   unknown 0x61 
   2155 18.974775   192.168.12.66         192.168.12.23         IPA      1514   unknown 0x87 
   2156 18.974788   192.168.12.66         192.168.12.23         IPA      520    OSMO EXT unknown 0xe6 
   2157 18.975679   192.168.12.23         192.168.12.66         TCP      66     5000 → 42522 [ACK] Seq=1438445295 Ack=4044280163 Win=65024 Len=0 TSval=2689562946 TSecr=3712487606
   2158 18.975684   192.168.12.23         192.168.12.66         TCP      66     5000 → 42522 [ACK] Seq=1438445295 Ack=4044285955 Win=61312 Len=0 TSval=2689562946 TSecr=3712487606
   2159 18.975688   192.168.12.23         192.168.12.66         TCP      66     5000 → 42522 [ACK] Seq=1438445295 Ack=4044287403 Win=60288 Len=0 TSval=2689562946 TSecr=3712487606
   2160 18.975693   192.168.12.23         192.168.12.66         TCP      66     5000 → 42522 [ACK] Seq=1438445295 Ack=4044288851 Win=59392 Len=0 TSval=2689562946 TSecr=3712487606
   2161 18.975920   192.168.12.23         192.168.12.66         TCP      66     5000 → 42522 [ACK] Seq=1438445295 Ack=4044290299 Win=58368 Len=0 TSval=2689562946 TSecr=3712487606
   2162 18.975924   192.168.12.23         192.168.12.66         TCP      66     5000 → 42522 [ACK] Seq=1438445295 Ack=4044291747 Win=57472 Len=0 TSval=2689562946 TSecr=3712487606
   2163 18.975929   192.168.12.23         192.168.12.66         TCP      66     5000 → 42522 [ACK] Seq=1438445295 Ack=4044292201 Win=57088 Len=0 TSval=2689562946 TSecr=3712487606
   2167 19.161365   192.168.12.23         192.168.12.66         IPA      83     unknown 0x54 
   2168 19.161501   192.168.12.66         192.168.12.23         TCP      66     42522 → 5000 [ACK] Seq=4044292201 Ack=1438445312 Win=64256 Len=0 TSval=3712487793 TSecr=2689563131
   2169 19.161369   192.168.12.23         192.168.12.66         IPA      434    unknown 0x6e 
   2171 19.163987   192.168.12.66         192.168.12.23         TCP      66     42522 → 5000 [FIN, ACK] Seq=4044292201 Ack=1438445681 Win=64128 Len=0 TSval=3712487796 TSecr=2689563131
   2172 19.165384   192.168.12.23         192.168.12.66         TCP      66     5000 → 42522 [ACK] Seq=1438445681 Ack=4044292202 Win=64128 Len=0 TSval=2689563135 TSecr=3712487796
\end{Verbatim}

\subsubsection{ドメイン名をIPアドレスに変換}
ログでこの手続きを行っているのは、以下の部分である。
 2140 18.969328   192.168.12.66         10.144.0.1            DNS      92     Standard query 0x3ba2 A exp1.rnea.comm.eng.osaka-u.ac.jp
 2142 18.970377   10.144.0.1            192.168.12.66         DNS      108    Standard query response 0x3ba2 A exp1.rnea.comm.eng.osaka-u.ac.jp A 192.168.12.23

 次に、No.2140のパケットキャプチャログの詳細を抜粋して示す。
User Datagram Protocol, Src Port: 34238, Dst Port: 53
    Source Port: 34238
    Destination Port: 53
    Length: 58
(中略)
    Queries
        exp1.rnea.comm.eng.osaka-u.ac.jp: type A, class IN
            Name: exp1.rnea.comm.eng.osaka-u.ac.jp
            [Name Length: 32]
            [Label Count: 7]
            Type: A (Host Address) (1)
            Class: IN (0x0001)

 同様に、No.2142のパケットキャプチャログの詳細を抜粋して示す。
User Datagram Protocol, Src Port: 53, Dst Port: 34238
    Source Port: 53
    Destination Port: 34238
    Length: 74
(中略)
    Queries
        exp1.rnea.comm.eng.osaka-u.ac.jp: type A, class IN
            Name: exp1.rnea.comm.eng.osaka-u.ac.jp
            [Name Length: 32]
            [Label Count: 7]
            Type: A (Host Address) (1)
            Class: IN (0x0001)
    Answers
        exp1.rnea.comm.eng.osaka-u.ac.jp: type A, class IN, addr 192.168.12.23
            Name: exp1.rnea.comm.eng.osaka-u.ac.jp
            Type: A (Host Address) (1)
            Class: IN (0x0001)
            Time to live: 2525 (42 minutes, 5 seconds)
            Data length: 4
            Address: 192.168.12.23

分析結果を表22にまとめた。
\begin{table}[h]
\centering
\caption{ドメイン名をIPアドレスに変換する手続きの分析}
\begin{tabular}{|c|c|c|c|c|c|}
\hline
No. & 送信元IPアドレス & 受信先IPアドレス & 送信元ポート番号 & 受信先ポート番号 & UDPセグメント長 \\
\hline
2140 & 192.168.12.66 & 10.144.0.1 & 34238 & 53 & 58 \\
2142 & 10.144.0.1 & 192.168.12.66 & 53 & 34238 & 74 \\
\hline
\end{tabular}
\end{table}

アプリケーション層のメッセージの内容を以下に示す。
2140 : exp1.rnea.comm.eng.osaka-u.ac.jpのWebページを保持するIPアドレスの問い合わせ
2142 : exp1.rnea.comm.eng.osaka-u.ac.jpのWebページを保持するIPアドレスを送信

 以上より、この2つのパケットの通信によって、クライアントは入力されたWebサーバに対応するIPアドレス(192.168.12.23)を取得することができることがわかる。

\subsubsection{TCPコネクションの確立}
 ログでこの手続きを行っているのは、以下の部分である。
  2144 18.972480   192.168.12.66         192.168.12.23         TCP      74     42522 → 5000 [SYN] Seq=4044280001 Win=64240 Len=0 MSS=1460 \texttt{SACK\_PERM}=1 TSval=3712487604 TSecr=0 WS=128
  2145 18.974209   192.168.12.23         192.168.12.66         TCP      74     5000 → 42522 [SYN, ACK] Seq=1438445294 Ack=4044280002 Win=65160 Len=0 MSS=1460 \texttt{SACK\_PERM}=1 TSval=2689562944 TSecr=3712487604 WS=128
  2146 18.974294   192.168.12.66         192.168.12.23         TCP      66     42522 → 5000 [ACK] Seq=4044280002 Ack=1438445295 Win=64256 Len=0 TSval=3712487606 TSecr=2689562944

ログから読み取ることができる情報を表23、表24にまとめる。
\begin{table}[h]
\centering
\caption{TCPコネクションを確立する手続きの分析1}
\begin{tabular}{|c|c|c|c|c|c|}
\hline
No. & 送信元IPアドレス & 受信先IPアドレス & 送信元ポート番号 & 受信先ポート番号 & TCPセグメント長 \\
\hline
2144 & 192.168.12.66 & 192.168.12.23 & 42522 & 5000 & 0 \\
2145 & 192.168.12.23 & 192.168.12.66 & 5000 & 42522 & 0 \\
2146 & 192.168.12.66 & 192.168.12.23 & 42522 & 5000 & 0 \\
\hline
\end{tabular}
\end{table}

\begin{table}[h]
\centering
\caption{TCPコネクションを確立する手続きの分析2}
\begin{tabular}{|c|c|c|c|}
\hline
No. & ONになっているTCPフラグ & シーケンス番号 & ACK番号 \\
\hline
2144 & SYN & 4044280001 & 0 \\
2145 & SYN, ACK & 1438445294 & 4044280002 \\
2146 & ACK & 4044280002 & 1438445295 \\
\hline
\end{tabular}
\end{table}

 実験課題2、実験課題3、実験課題4と同様の手順でTCPコネクションが確立されているとわかる。この3つのパケットにはアプリケーション層のメッセージが存在しない。

\subsubsection{HTTPメッセージの送受信}
 ログでこの手続きを行っているのは、以下の部分である。
\begin{Verbatim}[breaklines=true]
 2147 18.974476   192.168.12.66         192.168.12.23         IPA      227    unknown 0x53 
 2148 18.974618   192.168.12.66         192.168.12.23         IPA      1514   unknown 0x4e 
 2149 18.974640   192.168.12.66         192.168.12.23         IPA      1514   unknown 0x6e 
 2150 18.974654   192.168.12.66         192.168.12.23         IPA      1514   unknown 0xbc 
 2151 18.974667   192.168.12.66         192.168.12.23         IPA      1514   unknown 0x23 
 2152 18.974681   192.168.12.66         192.168.12.23         IPA      1514   unknown 0xb5 
 2153 18.974734   192.168.12.66         192.168.12.23         RSL      1514   ip.access PDCH DEACTIVATION 
 2154 18.974761   192.168.12.66         192.168.12.23         IPA      1514   unknown 0x61 
 2155 18.974775   192.168.12.66         192.168.12.23         IPA      1514   unknown 0x87 
 2156 18.974788   192.168.12.66         192.168.12.23         IPA      520    OSMO EXT unknown 0xe6 
 2157 18.975679   192.168.12.23         192.168.12.66         TCP      66     5000 → 42522 [ACK] Seq=1438445295 Ack=4044280163 Win=65024 Len=0 TSval=2689562946 TSecr=3712487606
 2158 18.975684   192.168.12.23         192.168.12.66         TCP      66     5000 → 42522 [ACK] Seq=1438445295 Ack=4044285955 Win=61312 Len=0 TSval=2689562946 TSecr=3712487606
 2159 18.975688   192.168.12.23         192.168.12.66         TCP      66     5000 → 42522 [ACK] Seq=1438445295 Ack=4044287403 Win=60288 Len=0 TSval=2689562946 TSecr=3712487606
 2160 18.975693   192.168.12.23         192.168.12.66         TCP      66     5000 → 42522 [ACK] Seq=1438445295 Ack=4044288851 Win=59392 Len=0 TSval=2689562946 TSecr=3712487606
 2161 18.975920   192.168.12.23         192.168.12.66         TCP      66     5000 → 42522 [ACK] Seq=1438445295 Ack=4044290299 Win=58368 Len=0 TSval=2689562946 TSecr=3712487606
 2162 18.975924   192.168.12.23         192.168.12.66         TCP      66     5000 → 42522 [ACK] Seq=1438445295 Ack=4044291747 Win=57472 Len=0 TSval=2689562946 TSecr=3712487606
 2163 18.975929   192.168.12.23         192.168.12.66         TCP      66     5000 → 42522 [ACK] Seq=1438445295 Ack=4044292201 Win=57088 Len=0 TSval=2689562946 TSecr=3712487606
 2167 19.161365   192.168.12.23         192.168.12.66         IPA      83     unknown 0x54 
 2168 19.161501   192.168.12.66         192.168.12.23         TCP      66     42522 → 5000 [ACK] Seq=4044292201 Ack=1438445312 Win=64256 Len=0 TSval=3712487793 TSecr=2689563131
\end{Verbatim}

次に、それぞれのパケットキャプチャログの詳細を抜粋して示す。
\begin{Verbatim}[breaklines=true]
・2147
Transmission Control Protocol, Src Port: 42522, Dst Port: 5000, Seq: 4044280002, Ack: 1438445295, Len: 161 
(中略)
    Flags: 0x018 (PSH, ACK)
(中略)
IPA protocol ip.access, type: unknown 0x53
    DataLen: 20559
    Protocol: Unknown (0x53)

・2148
Transmission Control Protocol, Src Port: 42522, Dst Port: 5000, Seq: 4044280163, Ack: 1438445295, Len: 1448 
(中略)
    Flags: 0x010 (ACK)
(中略)
IPA protocol ip.access, type: unknown 0x4e
    DataLen: 35152
    Protocol: Unknown (0x4e)

・2149
Transmission Control Protocol, Src Port: 42522, Dst Port: 5000, Seq: 4044281611, Ack: 1438445295, Len: 1448
(中略)
    Flags: 0x010 (ACK)
(中略)
IPA protocol ip.access, type: unknown 0x6e
    DataLen: 56045
    Protocol: Unknown (0x6e)

・2150
Transmission Control Protocol, Src Port: 42522, Dst Port: 5000, Seq: 4044283059, Ack: 1438445295, Len: 1448
(中略)
    Flags: 0x010 (ACK)
(中略)
IPA protocol ip.access, type: unknown 0xbc
    DataLen: 2298
    Protocol: Unknown (0xbc)

・2151
Transmission Control Protocol, Src Port: 42522, Dst Port: 5000, Seq: 4044284507, Ack: 1438445295, Len: 1448 
(中略)
    Flags: 0x010 (ACK)
(中略)
IPA protocol ip.access, type: unknown 0x23
    DataLen: 2368
    Protocol: Unknown (0x23)

・2152
Transmission Control Protocol, Src Port: 42522, Dst Port: 5000, Seq: 4044285955, Ack: 1438445295, Len: 1448
(中略)
    Flags: 0x018 (PSH, ACK)
(中略)
IPA protocol ip.access, type: unknown 0xb5
    DataLen: 55784
    Protocol: Unknown (0xb5)

・2153
Transmission Control Protocol, Src Port: 42522, Dst Port: 5000, Seq: 4044287403, Ack: 1438445295, Len: 1448 
(中略)
    Flags: 0x010 (ACK)
(中略)
IPA protocol ip.access, type: unknown 0x14
    DataLen: 3900
    Protocol: Unknown (0x14)
・2154
Transmission Control Protocol, Src Port: 42522, Dst Port: 5000, Seq: 4044288851, Ack: 1438445295, Len: 1448
(中略)
    Flags: 0x010 (ACK)
(中略)
IPA protocol ip.access, type: unknown 0x61
    DataLen: 32247
    Protocol: Unknown (0x61)

・2155
Transmission Control Protocol, Src Port: 42522, Dst Port: 5000, Seq: 4044290299, Ack: 1438445295, Len: 1448 
(中略)
    Flags: 0x010 (ACK)
(中略)
IPA protocol ip.access, type: unknown 0x87
    DataLen: 40278
    Protocol: Unknown (0x87)

・2156
Transmission Control Protocol, Src Port: 42522, Dst Port: 5000, Seq: 4044291747, Ack: 1438445295, Len: 454 
(中略)
    Flags: 0x018 (PSH, ACK)
(中略)
IPA protocol ip.access, type: OSMO EXT
    DataLen: 40686
    Protocol: OSMO EXT (0xee)
    Osmo ext protocol: Unknown (0xe6)

・2167
Transmission Control Protocol, Src Port: 5000, Dst Port: 42522, Seq: 1438445295, Ack: 4044292201, Len: 17 
(中略)
    Flags: 0x018 (PSH, ACK)
(中略)
IPA protocol ip.access, type: unknown 0x54
    DataLen: 18516
    Protocol: Unknown (0x54)
\end{Verbatim}

 このログから読み取ることができる各パケットの情報を表25と表26にまとめる。
\begin{table}[h]
\centering
\caption{HTTPメッセージの送受信の手続きの分析1}
\begin{tabular}{|c|c|c|c|c|c|}
\hline
No. & 送信元IPアドレス & 受信先IPアドレス & 送信元ポート番号 & 受信先ポート番号 & TCPセグメント長 \\
\hline
2147 & 192.168.12.66 & 192.168.12.23 & 42522 & 5000 & 161 \\
2148 & 192.168.12.66 & 192.168.12.23 & 42522 & 5000 & 1448 \\
2149 & 192.168.12.66 & 192.168.12.23 & 42522 & 5000 & 1448 \\
2150 & 192.168.12.66 & 192.168.12.23 & 42522 & 5000 & 1448 \\
2151 & 192.168.12.66 & 192.168.12.23 & 42522 & 5000 & 1448 \\
2152 & 192.168.12.66 & 192.168.12.23 & 42522 & 5000 & 1448 \\
2153 & 192.168.12.66 & 192.168.12.23 & 42522 & 5000 & 1448 \\
2154 & 192.168.12.66 & 192.168.12.23 & 42522 & 5000 & 1448 \\
2155 & 192.168.12.66 & 192.168.12.23 & 42522 & 5000 & 1448 \\
2156 & 192.168.12.66 & 192.168.12.23 & 42522 & 5000 & 454 \\
2157 & 192.168.12.23 & 192.168.12.66 & 5000 & 42522 & 0 \\
2158 & 192.168.12.23 & 192.168.12.66 & 5000 & 42522 & 0 \\
2159 & 192.168.12.23 & 192.168.12.66 & 5000 & 42522 & 0 \\
2160 & 192.168.12.23 & 192.168.12.66 & 5000 & 42522 & 0 \\
2161 & 192.168.12.23 & 192.168.12.66 & 5000 & 42522 & 0 \\
2162 & 192.168.12.23 & 192.168.12.66 & 5000 & 42522 & 0 \\
2163 & 192.168.12.23 & 192.168.12.66 & 5000 & 42522 & 0 \\
2167 & 192.168.12.23 & 192.168.12.66 & 5000 & 42522 & 17 \\
2168 & 192.168.12.66 & 192.168.12.23 & 42522 & 5000 & 0 \\
\hline
\end{tabular}
\end{table}

\begin{table}[h]
\centering
\caption{HTTPメッセージの送受信の手続きの分析2}
\begin{tabular}{|c|c|c|c|}
\hline
No. & ONになっているTCPフラグ & シーケンス番号 & ACK番号 \\
\hline
2147 & PSH, ACK & 4044280002 & 1438445295 \\
2148 & ACK & 4044280163 & 1438445295 \\
2149 & ACK & 4044281611 & 1438445295 \\
2150 & ACK & 4044283059 & 1438445295 \\
2151 & ACK & 4044284507 & 1438445295 \\
2152 & PSH, ACK & 4044285955 & 1438445295 \\
2153 & ACK & 4044287403 & 1438445295 \\
2154 & ACK & 4044288851 & 1438445295 \\
2155 & ACK & 4044290299 & 1438445295 \\
2156 & PSH, ACK & 4044291747 & 1438445295 \\
2157 & ACK & 1438445295 & 4044280163 \\
2158 & ACK & 1438445295 & 4044285955 \\
2159 & ACK & 1438445295 & 4044287403 \\
2160 & ACK & 1438445295 & 4044288851 \\
2161 & ACK & 1438445295 & 4044290299 \\
2162 & ACK & 1438445295 & 4044291747 \\
2163 & ACK & 1438445295 & 4044292201 \\
2167 & PSH, ACK & 1438445295 & 4044292201 \\
2168 & ACK & 4044292201 & 1438445312 \\
\hline
\end{tabular}
\end{table}
 
 No.2147〜No.2156のパケットではクライアントが連続してWebサーバへデータを送信しており、これはネットワークの輻輳によりACKが返ってこないからだと考えることができる。通常、クライアントは送信したデータに対するACKをまってからつぎの送信を行うが、ACKが帰ってこない場合は再送を行う仕組みとなっているため、クライアント側で再送が繰り返されたと考えられる。No.2157〜No.2163のパケットでは、Webサーバがクライアントに、ACKを連続で送信しており、これらのACK番号と、No.2147〜No.2156のシーケンス番号を照らし合わせると、一部のパケットに対してはACK応答が対応しているが、No.2148〜No.2150に対応するACKが存在しないことから、これらのパケットはネットワークの輻輳などによってサーバに届かなかった可能性が高い。No.2167ではwebサーバが画像の識別結果をクライアントに返しており、No.2168では、No.2167のパケットに対するACKが送信されている。これらから、通信が正常に再開され、アプリケーション層の処理が進行したことがわかる。

\subsubsection{TCPコネクションの終了}
 ログでこの手続きを行っているのは、以下の部分である。
\begin{Verbatim}[breaklines=true]
 2169 19.161369   192.168.12.23         192.168.12.66         IPA      434    unknown 0x6e 
 2171 19.163987   192.168.12.66         192.168.12.23         TCP      66     42522 → 5000 [FIN, ACK] Seq=4044292201 Ack=1438445681 Win=64128 Len=0 TSval=3712487796 TSecr=2689563131
 2172 19.165384   192.168.12.23         192.168.12.66         TCP      66     5000 → 42522 [ACK] Seq=1438445681 Ack=4044292202 Win=64128 Len=0 TSval=2689563135 TSecr=3712487796
\end{Verbatim}

 次に、No.2169のパケットキャプチャログの詳細を抜粋して示す。
\begin{Verbatim}[breaklines=true]
Transmission Control Protocol, Src Port: 5000, Dst Port: 42522, Seq: 1438445312, Ack: 4044292201, Len: 368 
(中略)
    Flags: 0x019 (FIN, PSH, ACK)
\end{Verbatim}

 このログから読み取ることができる各パケットの情報を表27と表28にまとめる。
\begin{table}[h]
\centering
\caption{TCPコネクションを終了する手続きの分析1}
\begin{tabular}{|c|c|c|c|c|c|}
\hline
No. & 送信元IPアドレス & 受信先IPアドレス & 送信元ポート番号 & 受信先ポート番号 & TCPセグメント長 \\
\hline
2169 & 192.168.12.23 & 192.168.12.66 & 5000 & 42522 & 368 \\
2171 & 192.168.12.66 & 192.168.12.23 & 42522 & 5000 & 0 \\
2172 & 192.168.12.23 & 192.168.12.66 & 5000 & 42522 & 0 \\
\hline
\end{tabular}
\end{table}

\begin{table}[h]
\centering
\caption{TCPコネクションを終了する手続きの分析2}
\begin{tabular}{|c|c|c|c|}
\hline
No. & ONになっているTCPフラグ & シーケンス番号 & ACK番号 \\
\hline
2169 & ACK, PSH, FIN & 1438445312 & 4044292201 \\
2171 & ACK, FIN & 4044292201 & 1438445681 \\
2172 & ACK & 1438445681 & 4044292202 \\
\hline
\end{tabular}
\end{table}
 
 No.2169のパケットでは、クライアントからの画像の識別結果が送信されると同時に、TCPコネクション終了要求であるFINが含まれている。これに対して、No. 2171のパケットではwebサーバがACKを返し、No.2172のパケットではwebサーバからのFINに対するACKが送信されており、正常なコネクションの終了処理が行われたことが確認できる。また、この3つのパケットにはいずれもアプリケーション層のメッセージは存在しない。

\subsection{実験課題7}
 送出間隔$T$を変化させながら、以下のコマンドを実行した。
\begin{Verbatim}[breaklines=true]
$ python3 client.py –num_images 100 –interval T –show_statistics
\end{Verbatim}

(a)送出間隔を変化させた時の画像のインデックスと遅延時間の関係グラフ
 送出間隔$T= 0.05, 0.1, 0.2 ,0.5 ,1.0$[s]のときのグラフを重ねてプロットしたら、図5のようになった。

\begin{figure}[H]
\centering
\includegraphics[width=0.8\textwidth]{images/図5.png}
\caption{画像のインデックスと遅延時間の関係グラフ}
\end{figure}

送出間隔$T$の値に関わらず比例関係を示し、$T$が大きくなるほど傾きが小さくなった。特に$T=0.2 ,0.5 ,1.0$のグラフの傾きはほぼ0となり、$T$が小さいと負荷が増すことで遅延時間が大きくなると考えられる。

(b)画像送出頻度と平均遅延時間の関係グラフ
 送出間隔$T$を$0.05-0.2$(0.01刻み)、$0.3-1.0$(0.1刻み)について、横軸を画像送出頻度、縦軸を平均遅延時間としてプロットした。

\begin{figure}[H]
\centering
\includegraphics[width=0.8\textwidth]{images/図6.png}
\caption{画像送出頻度と平均遅延時間の関係グラフ}
\end{figure}

 画像送出頻度が5よりも大きい場合は常に平均遅延時間が増加していく。これは(a)のグラフによって得られた結果と一致している。画像送出頻度が5以下、すなわち$T$が0.2以上の時に平均遅延時間がほぼ0となっていることがわかる。

(c)サーバの最大処理能力の推定
 3.1の理論でまとめたように、サーバの処理能力が到着ジョブの負荷を下回ると、滞留ジョブが増加し、平均遅延時間が増大する。図6より、画像送出頻度が5枚/秒を超えると遅延が顕著に増えるため、サーバの最大処理能力はおよそ5[枚/秒]であると推定できる。

(d)平均遅延時間と送出間隔$T$の和が最小となるような画像送出頻度の特定
  横軸を画像送出頻度、縦軸を平均時間と送出間隔の和としてプロットしたら図7のようになった。

\begin{figure}[H]
\centering
\includegraphics[width=0.8\textwidth]{images/図7.png}
\caption{画像送出頻度、平均遅延時間と送出間隔の和の関係グラフ}
\end{figure}
 このグラフより、平均遅延時間と送出間隔の和が最小となるのは$T=0.19$[s]の時である。この時の画像送出頻度は5.263[/s]である。

\section*{感想}
 情報通信基礎2の授業や個人でのwebアプリケーション開発においてpost manを利用するなど、TCPの3-ウェイハンドシェイクの流れやHTTPメソッドのリスエスト/レスポンスについて、概念として抽象的には抑えることができていたが、今回の実験のように実際にパケットの流れを可視化してログを読むということは初めてだった。これまで、HTTPメソッドは身近なもので、TCPについては学問として学んだものといったイメージが強く、結びつけることができていなかった。今回の実験を通して、普段なんとなく利用しているDNSや通信の仕組みについて、UDP, TCP, HTTPの基本動作、通信トラヒック工学の基礎について学びを深めることができた。
\section*{参考文献}
・電子情報工学専門実験C1 通信ネットワーク工学の基礎
・Wiresharkで通信プロトコルを見る
\url{https://future-architect.github.io/articles/20210823b/}
・GPU推論サーバの待ち行列モデル 滝根研究室
\url{http://www2b.comm.eng.osaka-u.ac.jp/~yoshiaki/slides/pdf/2019_q-symp.pdf}
・5章 待ち行列ネットワークモデル 電子情報通信学会知識ベース
\url{https://www.ieice-hbkb.org/files/05/05gun_01hen_05.pdf}

\end{document}
