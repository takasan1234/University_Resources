\documentclass[a4paper,11pt]{jsarticle}

\usepackage{plautopatch}
\usepackage{amsmath}
\usepackage{amsfonts}
\usepackage{amssymb}
\usepackage[dvipdfmx]{graphicx}
\usepackage[dvipdfmx]{color} % dvipdfmxドライバーを明示的に指定
\DeclareGraphicsExtensions{.pdf,.png,.jpg}
\usepackage{xurl}
\usepackage{url}
\urlstyle{tt}
\usepackage{bm}
\usepackage{listings}
\usepackage{color}
\usepackage{float}
\usepackage[justification=centering]{caption}

% ページ設定
\usepackage{geometry}
\geometry{margin=25mm} % Set margin to 25mm
\setlength{\parindent}{0pt} % 段落のインデントを0に設定

\begin{document}

% タイトルページ
\begin{titlepage}
\centering
\vspace*{2cm}

{\Large 電子情報工学科}

\vspace{1cm}

{\large 情報通信工学専門実験 報告}

\vspace{0.5cm}

{\large 第 1 号}

\vspace{2cm}

{\large 実 験 題 目}

\vspace{0.5cm}

{\large C2 電子回路の基礎実験}

\vspace{1cm}

情報システム工学コース 第2班

\vspace{0.5cm}

辻 孝弥

\vspace{2cm}

{\large 報 告 者}

\vspace{0.5cm}

08D23091番 辻 孝弥(情報システム工学コース)

\vspace{0.3cm}

電子メールアドレス:u316705e@ecs.osaka-u.ac.jp

\vspace{2cm}

令和8年 1月12日

\vspace{1cm}

大阪大学工学部電子情報工学科

\end{titlepage}

\newpage

% 目的
\section*{目的}

トランジスタを用いる増幅回路の基本挙動を把握し、バイポーラトランジスタによるエミッタ接地回路の増幅特性を測定・確認する。得られた知見を踏まえて、要求仕様を満たすエミッタ接地増幅回路を設計・試作し、性能を検証する。

% 1. 実験日時
\section{実験日時}

第一回:2025年12月18日

第二回:2025年12月25日

% 2. 理論
\section{理論}

バイポーラトランジスタを用いたエミッタ接地回路の増幅特性を、図1の回路図に沿って整理する。

\begin{figure}[H]
\centering
\includegraphics[width=0.8\textwidth]{images/図1.png}
\caption{バイポーラトランジスタを用いたエミッタ接地回路の回路図}
\end{figure}

まず直流動作に着目する。ベース端子に流れ込む電流を無視できると仮定すると、ベース電圧 $V_B$ は次式で表される。

\begin{equation}
V_B = \frac{R_2}{R_1 + R_2} V_{CC} \tag{1}
\end{equation}

エミッタ電圧 $V_E$ はベース電圧から順方向電圧降下 $V_{BE}$ を差し引いた値となり、次の関係で与えられる。

\begin{equation}
V_E = V_B - V_{BE} = \frac{R_2}{R_1 + R_2} V_{CC} - V_{BE} \tag{2}
\end{equation}

トランジスタが導通するには $V_B > V_{BE}$ を満たす必要があり、この条件を踏まえて $R_1$ と $R_2$ を設定する。オン状態ではエミッタに直流電圧が生じ、エミッタ抵抗 $R_E$ に電流 $I_E$ が流れる。ベース電流を無視できるとすれば、$I_E$ はほぼコレクタ電流 $I_C$ と等しい。

\begin{equation}
I_E = \frac{V_E}{R_E} = \frac{V_B - V_{BE}}{R_E} \approx I_C \tag{3}
\end{equation}

このときコレクタ端子の直流電圧成分 $V_C$ は次式となる。

\begin{equation}
V_C = V_{CC} - R_C I_C \approx V_{CC} - \frac{R_C}{R_E} (V_B - V_{BE}) \tag{4}
\end{equation}

なお、トランジスタを適切な動作点に保つためには $V_C >> V_E$ を確保する必要があり、これを満たすように $V_E$ と $V_C$ を設定する。

次に入力ポートから交流電圧 $v_i$ を加えた場合を考える。交流成分はコンデンサ $C_1$ を介してベース電圧を変化させるため、その変化分を $\Delta v_B$ とおくと、交流を含むベース電圧 $v_B$ は次式となる。

\begin{equation}
v_B = V_B + \Delta v_B \tag{5}
\end{equation}

エミッタ電圧 $v_E$ も同様に、$v_B$ から一定の $V_{BE}$ を差し引いた形で表される。

\begin{equation}
v_E = v_B - V_{BE} = V_B + \Delta v_B - V_{BE} \tag{6}
\end{equation}

このときのコレクタ電圧 $v_C$ は次式で表される。

\begin{equation}
v_C = V_{CC} - \frac{R_C}{R_E} (V_B + \Delta v_B - V_{BE}) \tag{7}
\end{equation}

ここで $v_C$ を直流成分 $V_C$ と交流成分 $\Delta v_C$ に分けて $v_C = V_C + \Delta v_C$ とすると、$\Delta v_B$ と $\Delta v_C$ の関係は次のように書ける。

\begin{equation}
\Delta v_C = - \frac{R_C}{R_E} \Delta v_B \tag{8}
\end{equation}

ここで $C_1$ を十分大きく設定すれば、入力交流電圧 $v_i$ がほぼそのまま $\Delta v_B$ となり、$v_i = \Delta v_B$ とみなせる。同様に $C_2$ を適切に選ぶことで、コレクタの交流成分 $\Delta v_c$ がそのまま出力され、$v_o = \Delta v_c$ となる。従って、入出力の交流成分は次式で結ばれる。

\begin{equation}
v_o = - \frac{R_C}{R_E} v_i \tag{9}
\end{equation}

この式から電圧増幅率は $\frac{R_C}{R_E}$ であり、出力電圧が入力に対して反転することがわかる。

\begin{figure}[H]
\centering
\includegraphics[width=0.8\textwidth]{images/図2.png}
\caption{回路の(a) 入力側から見た場合の等価回路と(b) 出力側から見た場合の等価回路}
\end{figure}

続いて入力インピーダンスと出力インピーダンスを確認する。入力インピーダンスとは図2(a)に示すように、回路が外部から見せる内部インピーダンス $Z_{in}$ を指す。$Z_{in}$ が十分大きければ入力電圧を印加してもほとんど電流が流れず、逆に小さいと入力時に大きな電流が流れて外部回路の状態を乱してしまう。したがって $Z_{in}$ は大きいほど望ましい。

図1の入力インピーダンスを求めるにあたり、簡単化のため $C_1$ を無視し、ベースに電流 $\Delta i_B$ を直接与える場合を考える。ベース電流は十分小さいとみなせるため、入力部は図3(a)で近似できる。このときベース電圧は $V_B + \Delta v_B$ に変化し、$R_2$ に流れる電流 $i_2$ は次式となる。

\begin{equation}
i_2 = \frac{V_B + \Delta v_B}{R_2} \tag{10}
\end{equation}

一方、$R_1$ に流れる電流 $i_1$ は

\begin{equation}
i_1 = \frac{V_B + \Delta v_B}{R_1} - \Delta i_B \tag{11}
\end{equation}

となる。$V_{CC} = R_1 i_1 + R_2 i_2$ の関係を用いると、次式が得られる。

\begin{equation}
\frac{\Delta v_B}{\Delta i_B} = \frac{R_1 R_2}{R_1 + R_2} = R_{//} \tag{12}
\end{equation}

これが求める入力インピーダンスであり、入力側等価回路は図3(b)のように表せる。$Z_{in}$ を大きくするには、$R_1$ と $R_2$ に大きな抵抗値を選ぶことが有効となる。

\begin{figure}[H]
\centering
\includegraphics[width=0.8\textwidth]{images/図3.png}
\caption{エミッタ接地回路の(a)入力回路と(b)入力側からみた等価回路}
\end{figure}

出力インピーダンスは図2(b)のように、出力側に挿入された内部インピーダンス $Z_{out}$ とみなせる。負荷抵抗 $R_L$ を接続した際、$R_L$ が $Z_{out}$ より十分大きければ $Z_{out}$ による電圧降下はほぼ無視できるが、$R_L$ が同程度になると電圧降下が無視できなくなる。大きな電流を取り出したい場合、すなわち小さな負荷を接続する際は $Z_{out}$ を小さくすることが望ましい。ここで図1のエミッタ接地回路における $Z_{in}$ と $Z_{out}$ を求める。

続いて出力インピーダンス $Z_{out}$ を求める。図4(a)のように外部へ電流 $\Delta i_{out}$ を取り出すと、コレクタ電圧は $V_C$ から $\Delta v_{out}$ だけ変化すると考えられる。一方でエミッタ電流 $I_E$ は式(3)のまま外部電流に依存しない点に注意する。$R_C$ に流れる電流は $I_C + \Delta i_{out}$ となり、$V_C + \Delta v_{out} = V_{CC} - R_C \cdot (I_C + \Delta i_{out})$ が成り立つため、次式が得られる。

\begin{equation}
\Delta v_{out} = -R_C \Delta i_{out} \tag{13}
\end{equation}

これは外部に取り出した電流 $\Delta i_{out}$ に比例してコレクタ電圧が $R_C \cdot \Delta i_{out}$ だけ低下することを示している。ゆえに図4(b)の等価回路で表され、出力インピーダンスは $R_C$ と判断できる。

負荷抵抗が $R_C$ に近いと出力電圧は大きく低下する。電圧降下を抑えるには $R_C$ を小さくしたくなるが、式(9)から $R_C$ を下げれば同時に $R_E$ も小さくして増幅率を確保する必要が生じ、大きなエミッタ電流を流すことになる。バイポーラトランジスタではエミッタ電流が増えるとベース電流も増加し、設計したベース電圧を保つには $R_1$ と $R_2$ を小さくして電流を増やさざるを得ず、結果として入力インピーダンスが下がる。$R_1$、$R_2$、$R_E$、$R_C$ の選定にはトレードオフがあり、いずれかを極端に振ると別の仕様が悪化するため、全体のバランスを見て最適値を決める必要がある。

\begin{figure}[H]
\centering
\includegraphics[width=0.8\textwidth]{images/図4.png}
\caption{エミッタ接地回路の(a)出力回路と(b)出力側からみた等価回路}
\end{figure}

最後に入力側コンデンサ $C_1$ が交流成分に与える影響を確認する。この場合、図5に示すように $C_1$ と式(12)の $Z_{in}$ が直列に接続された等価回路となり、$Z_{in}$ 両端の電圧がベースの交流成分 $\Delta v_B$ となる。これは $CR$ 直列の高域通過フィルタとして扱える。入力交流電圧 $v_i$ の角周波数を $\omega$ とすると、$C_1$ のインピーダンスは $\frac{1}{j\omega C_1}$ であり、複素表示 $\dot{v}_i$ と $\dot{\Delta v_B}$ の関係は次のようになる。

\begin{equation}
\dot{\Delta v_B} = \frac{R_{//}}{R_{//} + \frac{1}{j\omega C_1}} \dot{v}_i \tag{14}
\end{equation}

となる。したがって、

\begin{equation}
|\dot{\Delta v_B}| = \frac{1}{\sqrt{1 + \left(\frac{1}{\omega C_1 R_{//}}\right)^2}} |\dot{v}_i| \tag{15}
\end{equation}

となり、$\omega$ が大きくなるほど $\Delta v_B$ の振幅が $v_i$ に近づく高域通過特性となる。$\Delta v_B$ が $v_i$ の $\frac{1}{\sqrt{2}}$ となる周波数、すなわち低域カットオフ周波数は次式で与えられる。

\begin{equation}
f_L = \frac{1}{2\pi C_1 R_{//}} \tag{16}
\end{equation}

で与えられ、この $f_L$ を低域カットオフ周波数と呼ぶ。入力信号の周波数帯に合わせて $R_1$、$R_2$、$C_1$ を選ぶ必要がある。なお式(12)までの議論では $C_1$ の影響を無視し $v_i = \Delta v_B$ としていた点に注意する。

\begin{figure}[H]
\centering
\includegraphics[width=0.8\textwidth]{images/図5.png}
\caption{エミッタ接地回路の入力コンデンサ$C_1$を考えたときの入力側等価回路}
\end{figure}

% 3. 実験結果
\section{実験結果}

本実験は2週にわたり実施した。1週目の内容と結果を\ref{subsec:amplification}に、2週目の内容と結果を\ref{subsec:design-task}にまとめる。

\subsection{バイポーラトランジスタを用いたエミッタ接地回路の増幅特性評価}
\label{subsec:amplification}

トランジスタ(NPN 型バイポーラトランジスタ 2SC1815)を用いたエミッタ接地増幅回路の回路図とブレッドボード写真を図6に示す。実機では(11)(12)で触れる可変抵抗を挿入している。以下の(1)〜(12)の手順で測定を行った。

\begin{figure}[H]
\centering
\includegraphics[width=0.8\textwidth]{images/図6.png}
\caption{バイポーラトランジスタを用いたエミッタ接地増幅回路の(a)回路図(b)ブレッドボード上の写真}
\end{figure}

(1) 電源電圧の設定

直流安定化電源を回路に接続し、電源電圧 $V_{CC}$ を 15V に調整した(ディジタルマルチメータで15Vを確認)。

(2) トランジスタのベース直流電圧 $V_B$ の測定

ディジタルマルチメータでベース電圧 $V_B$ を測定したところ 2.462V だった。$V_B$ は $V_{CC}$ と $R_1, R_2$ から式(1)で求まり、理論値は 2.50V となる。

(3) トランジスタのエミッタ直流電圧 $V_E$ の測定

エミッタ直流電圧 $V_E$ を測定すると 1.842V であった。ベースとの差 $V_{BE}$ は 2.462-1.842=0.620V となり、ベース電圧より約0.6〜0.7V低いことを確認した。

(4) トランジスタのコレクタ直流電圧 $V_C$ の測定

コレクタ直流電圧 $V_C$ を測ると 5.80V であった。式(4)より $V_C$ は $R_C, R_E, V_B, V_{BE}, V_{CC}$ で決まり、$V_B$ と $V_{BE}$ を $R_1,R_2$ で定めることを踏まえると理論値は 5.75V($V_{BE}=0.65V$ として計算)となる。

(5) 入力電圧波形の設定

ファンクションジェネレータを交流入力に接続し、周波数 1kHz・振幅 0.5V の正弦波を入力した。入力波形 $v_i$ をオシロスコープ CH1 に表示し、振幅が 0.5V になるよう出力を調整した結果、設定値は 480mVpp となった。

(6) ベース電圧波形 $v_B$ の測定

オシロスコープでベース電圧 $v_B$ の時間波形を測定し CH2 に表示した。CH1 の $v_i$ と比較すると、(2)で得た直流電圧が重畳していることがわかる。CH1 と CH2 を併記した様子を図7に示す。$v_B$ の極大値は 3.04V、極小値は 2.00V であった。

\begin{figure}[H]
\centering
\includegraphics[width=0.6\textwidth,natwidth=650,natheight=394]{images/図7.jpg}
\caption{入力AC電圧(黄線)とベース電圧(青線)のオシロスコープ波形}
\end{figure}

(7) エミッタ電圧波形 $v_E$ の測定

エミッタ電圧 $v_E$ の時間変化を測定し、CH1 に表示した結果を図8に示す。交流成分の形は変わらず、ベース電圧から(3)で得た $V_{BE}$ が差し引かれていることを確認した。$v_E$ の極大値は 2.36V、極小値は 1.40V であった。

\begin{figure}[H]
\centering
\includegraphics[width=0.6\textwidth,natwidth=644,natheight=385]{images/図8.jpg}
\caption{ベース電圧(青線)エミッタ電圧(黄色)のオシロスコープ波形}
\end{figure}

(8) コレクタ電圧波形 $v_C$ の測定

コレクタ電圧 $v_C$ を測定し CH2 に表示、CH1 には $v_i$ を表示した様子を図9に示す。$v_C$ の極大値は 8.32V、極小値は 3.52V であり、$v_i$ と比べて位相が反転している。これは式(8)で示すように $\Delta v_C$ が $\Delta v_B (=v_i)$ の $-\frac{R_C}{R_E}$ 倍となり、$\Delta v_C$ がそのまま $v_o$ となるためである。

\begin{figure}[H]
\centering
\includegraphics[width=0.6\textwidth,natwidth=586,natheight=354]{images/図9.jpg}
\caption{入力 AC 電圧(黄線)とコレクタ電圧(青線)のオシロスコープ波形}
\end{figure}

(9) 出力電圧波形 $v_o$ の測定

出力電圧 $v_o$ を測定し CH2 に表示した結果を図10に示す。図から、(8)で観測した $v_C$ から直流成分が除かれた波形になっていることがわかる。$v_i$ と比較すると振幅は約5倍に増幅されており、式(9)から電圧増幅率は $\frac{R_C}{R_E}$ と表される。

\begin{figure}[H]
\centering
\includegraphics[width=0.6\textwidth,natwidth=629,natheight=381]{images/図10.jpg}
\caption{入力AC電圧(黄線),出力AC電圧(青色)のオシロスコープ波形}
\end{figure}

(10) 周波数特性の測定

オシロスコープで CH1 に $v_i$、CH2 に $v_o$ を表示し、$v_i$ の振幅が常に 0.25V となるようジェネレータを調整しつつ 30Hz〜3MHz まで掃引した。得られた周波数—出力振幅の関係を図11に示す(横軸は対数目盛)。縦軸は 1kHz 時の $v_o$ を 0dB とした相対値である。低域で $v_o$ が −3dB となる周波数を低域カットオフ周波数と呼び、グラフから約91Hz と読めた。式(12)と式(16)より、この値は $R_1$、$R_2$、$C_1$ により決まる。

\begin{figure}[H]
\centering
\includegraphics[width=0.8\textwidth]{images/図11.png}
\caption{周波数[Hz](横軸)と出力電圧[dB](縦軸)の関係}
\end{figure}

(11) 入力インピーダンス $Z_{in}$ の測定

図12(a)のように、ファンクションジェネレータ(FG)と入力端子の間に可変抵抗を挿入する。つまみを反時計回りに回すと 0Ω となるため、(9)までの測定ではこの状態にしていた。図12(a)の2点の電圧をオシロスコープで観測し、まず可変抵抗を 0Ω にすると両者は一致する。続いて抵抗を増やし、CH2 の振幅が CH1 の半分になるよう調整し、可変抵抗を抜いてディジタルマルチメータで値を測定したところ 16.82$k\Omega$ であった。この値をエミッタ接地回路の $Z_{in}$ とみなす。図12(a)より CH2 が半分になるとき可変抵抗と $Z_{in}$ が同じ値になるため、この方法で $Z_{in}$ を求められる。また式(12)から $Z_{in}$ は $R_1$ と $R_2$ に依存することがわかる。

\begin{figure}[H]
\centering
\includegraphics[width=0.8\textwidth]{images/図12.png}
\caption{エミッタ接地回路の(a)入力インピーダンス測定系と(b)出力インピーダンス測定系}
\end{figure}

(12) 出力インピーダンス $Z_{out}$ の測定

(11)で外した入力側可変抵抗を元に戻し 0Ω とする。図12(b)の2点をオシロスコープで測定すると、CH2 には(9)と同じ $v_o$(1kHz、振幅は(9)と同値)が表示され、図10と同じ波形となる。(9)まででは出力側の可変抵抗は基板上にあるが GND へは接続しておらず、出力端は開放で可変抵抗を回しても $v_o$ は変化しない。この開放時の $v_o$ を基準として覚えておく。次に図12(b)のように可変抵抗のGND側リードをGNDに接続し負荷とすると、CH2 の $v_o$ は減少し、抵抗を最大にしても基準値を超えない。可変抵抗を調整して CH2 の振幅が基準の1/2になるようにし、抵抗を抜いて測定したところ 9.94$k\Omega$ であった。これを $Z_{out}$ とする。CH2 が半分になるとき $Z_{out}$ と可変抵抗値が等しくなるため、この手順で $Z_{out}$ を求められる。また式(13)から $Z_{out}$ は $R_C$ で決まる。

\subsection{バイポーラトランジスタを用いたエミッタ接地増幅回路の設計に関する課題}
\label{subsec:design-task}

\subsubsection{エミッタ接地増幅回路の設計}
\label{subsec:design}

以下の仕様を満たすエミッタ接地増幅回路を設計した。なお、交流結合用コンデンサ $C_1, C_2$ は 0.1$\mu F$ とした。

・ 電源電圧:5V

・ 周波数 1kHz・振幅 0.5V の正弦波入力に対する電圧増幅率:2.5倍以上

・ 入力インピーダンス:10$k\Omega$以上

・ 出力インピーダンス:5$k\Omega$以下

・ 低域カットオフ周波数: 500Hz以下

$R_1=61k\Omega, R_2=20k\Omega, R_C=3k\Omega, R_E=1k\Omega$として設計を行なった。

\subsubsection{設計回路の作製}

ブレッドボード上に\ref{subsec:design}で決めた回路を組み立てた。

\subsubsection{3.1(1)〜(4)と同様の手順で$V_B$、$V_E$、$V_C$を測定した結果は以下の表1のようになった。}

測定結果を理論値と照らし合わせると概ね一致していることが確認できた。理論値は $V_{BE}=0.65V$ として計算した。

\begin{table}[h]
\centering
\caption{ベース電圧、エミッタ電圧、コレクタ電圧の測定結果}
\begin{tabular}{lcc}
\hline
 & 理論値[V] & 測定値[V] \\
\hline
$V_B$ & 1.130 & 1.112 \\
$V_E$ & 0.510 & 0.527 \\
$V_C$ & 3.278 & 3.230 \\
\hline
\end{tabular}
\end{table}

\subsubsection{作製した回路の周波数 1kHz 振幅 0.5V の正弦波入力に対する電圧増幅率と出力波形を測定した結果を以下の図 13 に示す。}

増幅率は約 2.69 倍となり、\ref{subsec:design}の条件を満足していることを確認した。

\begin{figure}[H]
\centering
\includegraphics[width=0.6\textwidth,natwidth=800,natheight=480]{images/図13.jpg}
\caption{入力AC電圧(黄線),出力AC電圧(青色)のオシロスコープ波形}
\end{figure}

\subsubsection{コレクタ交流電圧の極小値およびエミッタ交流電圧の極大値の測定値は、コレクタ交流電圧の極小値は 1.84V,エミッタ交流電圧の極大値は1.12Vとなった。}

両者の差は 0.72V であり、今回用いたトランジスタの「コレクタ・エミッタ間飽和電圧」(約0.2V)を十分上回っていることを確認した。

\subsubsection{3.1(10)〜(12)と同様に低域カットオフ周波数、入力インピーダンス、出力インピーダンスを測定した。}

低域カットオフ周波数は約90.2Hz となり、条件を満たすことを確認した。$Z_{in}=17.14k\Omega$、$Z_{out}=3.00k\Omega$を得た。

\begin{figure}[H]
\centering
\includegraphics[width=0.8\textwidth]{images/図14.png}
\caption{周波数[Hz](横軸)と出力電圧[dB](縦軸)の関係}
\end{figure}

% 4. 考察
\section{実験に対する考察}

\subsection{実験項目内の質問}

質問事項は3.1でも触れているが、ここで簡潔に整理する。

・3.1.(2)について

式(1)より、ベース電圧 $V_B$ は抵抗 $R_1, R_2$ によって決まる。

・3.1.(4)について

式(4)から、$V_C$ は $R_C, R_E, R_1, R_2$ に依存する。

・3.1.(8)について

式(8)より $\Delta v_C$ は $\Delta v_B (=v_i)$ の $-\frac{R_C}{R_E}$ 倍となるため位相が反転する。

・3.1.(9)について

式(9)より電圧増幅率は $\frac{R_C}{R_E}$ となり、3.1.(10)の式(12)と式(16)から低域カットオフ周波数は $R_1$、$R_2$、$C_1$ で決まる。

・3.1.(11)について

CH2の振幅がCH1の半分になった時点で $Z_{in}$ と可変抵抗値が一致するとみなし、この方法で $Z_{in}$ を測定できた。また式(12)から $Z_{in}$ は $R_1$ と $R_2$ に依存する。

・3.1.(12)について

CH2 の振幅が(9)の半分になったとき $Z_{out}$ と可変抵抗が同じとみなせ、この方法で $Z_{out}$ を求められる。式(13)より $Z_{out}$ は $R_C$ で決まる。

\subsection{$V_{BE}$が約0.7Vである理由}

トランジスタはベース・エミッタ間電圧が一定値を超えるとONになるが、その値は材料によって異なる。今回使用した NPN 型バイポーラトランジスタ 2SC1815 はシリコン製であり、シリコントランジスタでは $V_{BE}$ が 0.6〜0.8V 程度になることが知られている。3.1(3)で 0.616V が得られたのはこの特性によると考えられる。

\subsection{電圧増幅率の式の導出}

今回は $I_E \approx I_C$ と近似して計算した(実際にはベース電流が存在し、コレクタ電流の1/$h_{FE}$ が $V_{CC}$ から $R_1$ を経てベースへ流れ込むため、エミッタ電流からその分を引いたものがコレクタ電流となる)が、正しくは次のように表される($h_{FE}$:直流電流増幅率)。

\begin{equation*}
I_C = I_E - I_B = I_E - \frac{I_E}{h_{FE}}
\end{equation*}

これを整理すると式(17)のようになる。

\begin{equation}
I_C = \frac{h_{FE}}{h_{FE} + 1} I_E \tag{17}
\end{equation}

ここで2で示した式(3)より、$I_E$ は次のように書ける。

\begin{equation}
I_E = \frac{V_B - V_{BE}}{R_E} \tag{3}
\end{equation}

式(3)と式(17)を組み合わせると $I_C$ は次式となる。

\begin{equation*}
I_C = \frac{h_{FE} (V_B - V_{BE})}{(h_{FE} + 1) R_E}
\end{equation*}

これと式(4)より

\begin{equation}
V_C = V_{CC} - R_C I_C = V_{CC} - \frac{R_C h_{FE}}{R_E (1 + h_{FE})} (V_B - V_{BE}) \tag{18}
\end{equation}

となる。交流成分も含めると次のように表せる。

\begin{equation}
V_C + \Delta v_C = V_{CC} - \frac{R_C \cdot h_{FE}}{R_E (1 + h_{FE})} (V_B + \Delta v_B - V_{BE}) \tag{19}
\end{equation}

式(19)−式(18)より $\Delta v_C$ は

\begin{equation}
\Delta v_C = -\frac{R_C}{R_E} \cdot \frac{h_{FE}}{1 + h_{FE}} \Delta v_B \tag{20}
\end{equation}

式(20)から $h_{FE}$ を含んだ電圧増幅率は

\begin{equation*}
\frac{R_C}{R_E} \cdot \frac{h_{FE}}{1 + h_{FE}}
\end{equation*}

となる。従って実際の交流電圧増幅率は $\frac{R_C}{R_E}$ より小さくなるはずである。3.1(9)の結果が 4.8 倍と理論値 $\frac{R_C}{R_E} = 5$ を下回ったのは $h_{FE}$ の影響と考えられる。

\subsection{低域カットオフ周波数}

\subsubsection{方法 1}

式(15)で $v_i$ の係数が $1/\sqrt{2}$ となる周波数が、式(16)で表される低域カットオフ周波数に相当する。

\begin{equation}
f _ {L} = \frac {1}{2 \pi C _ {1} R _ {/ /}} \tag{16}
\end{equation}

$C_1 = 0.1[\mu F], R_{/ /} = \frac{100 \cdot 20}{100 + 20} = 16.67[k\Omega]$ を代入すると

\begin{equation*}
f _ {L} = \frac {1}{2 \pi C _ {1} R _ {/ /}} \approx 95.5[Hz]
\end{equation*}

\subsubsection{方法 2}

式(15)の係数を $1\mathrm{kHz}$ で求め、そこに $1/\sqrt{2}$ を掛けた値になる周波数を逆算することで $f_L$ を求める。$1\mathrm{kHz}$ では角周波数 $\omega$ は $2000\pi$ となり、$C_1 = 0.1[\mu F], R_{/ /} = \frac{100 \cdot 20}{100 + 20} = 16.67[k\Omega]$ を代入して計算する。

\begin{equation*}
\frac {1}{\sqrt {1 + \left\{\frac {1}{(2000\pi \cdot (0.1 \times 10^{-6}) \cdot (16.67 \times 10^{3})}\right\}^2}} \approx 0.995
\end{equation*}

これを $1/\sqrt{2}$ 倍すると約 0.704 となるため、次の式を満たす $f_L$ を求める。

\begin{equation*}
\frac {1}{\sqrt {1 + \left(\frac {1}{2\pi C _ {1} R _ {/ /} f _ {L}}\right) ^ {2}}} = 0.704
\end{equation*}

計算するとおよそ 94.6 となる。

これら2つの算出値は3.1(10)で得た低域カットオフ周波数の測定値とほぼ一致しており、差は0.9Hz程度である。

\subsection{高域カットオフ周波数}

高域カットオフ周波数は、高周波域で利得が維持できなくなる境界周波数である。ここでは $R_{//}$ に加えてベース–エミッタ間容量 $C_{be}$、ベース–コレクタ間容量 $C_{bc}$、増幅率 $A$ が影響する。特にベース–コレクタ間容量はミラー効果により見かけが $(1 + A)$ 倍になる。また今回はエミッタ抵抗を持つため、増幅率 $A$ は次式で近似できる。

\begin{equation*}
A = \frac {R _ {c}}{R _ {E}}
\end{equation*}

増幅回路の高域カットオフ周波数は、抵抗と容量による一次ローパス特性の −3dB 点として求められる。

実効容量

\begin{equation*}
C _ {in} = C _ {BE} + \left(1 + \frac {R _ {c}}{R _ {E}}\right) C _ {BC}
\end{equation*}

と、入力側で有効な並列抵抗 $R_{||}$ から成る高周波ローパスフィルタとみなせる。

以上より、高域カットオフ周波数は次式で表される。

\begin{equation*}
f _ {H} = \frac {1}{2 \pi R _ {//} (C _ {be} + (1 + \frac {R _ {c}}{R _ {E}}) C _ {bc})}
\end{equation*}

% 5. 参考文献
\section{参考文献}

[1] 電子回路の基礎実験(c2)実験テキスト

[2] dB値の計算方法
\url{https://detail-infomation.com/voltage-power-decibel-db/}

\end{document}
