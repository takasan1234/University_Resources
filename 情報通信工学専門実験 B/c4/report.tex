\documentclass{jlreq}
% \usepackage{xeCJK}
\usepackage[top=2.5cm, bottom=2.5cm, left=2.5cm, right=2.5cm]{geometry}

\pagestyle{empty}


\begin{document}

\begin{center}
    {\huge 電子情報工学科} \\
    \vspace{0.7cm}
    {\huge     工学専門実験  報告} \\ %科目名を記入
    \vspace{1.4cm}
    {\Large 第 1 号} \\ %実験番号を記入
    \vspace{2.1cm}
    {\large 実  験  題  目} \\
    \vspace{1.5cm}
    \underline{\huge C4 信号変調の基礎実験} \\%実験題目を記入
    \vspace{0.7cm}
    \underline{情報システム工学コース   第 2 班} \\%コース名と班を記入
    \underline{\Large 阿部 寛希}\\%同じ班の氏名を記入
    \underline{\Large 尾本 涼太}\\%同じ班の氏名を記入
    \underline{\Large 菊池 圭祐}\\%同じ班の氏名を記入
\end{center}

\vfill

\begin{center}
    {\Large 報 告 者} \\
    \vspace{1cm}
    \underline{\large 08D23091 番 辻孝弥 (情報システム工学コース)} \\%報告者の学籍番号、氏名、コース名を記入
    \vspace{1cm}
    \underline{\large 電子メールアドレス:\hspace{1cm}u316705e@ecs.osaka-u.ac.jp }\\%報告者のメールアドレスを記入
\end{center}

\vfill

\begin{center}
    \large 令和 7 年 11 月 19 日 \\%報告日を記入
    \vspace{0.7cm}
    \Large 大阪大学工学部電子情報工学科
\end{center}

\newpage
\pagestyle{plain}
\pagenumbering{arabic}

\section{目的}
本実験の目的は、ベースバンド信号を振幅変調および位相変調した際の、
時間波形と周波数スペクトルの特徴を理解することである。
これにより、信号を時間領域と周波数領域の対応関係で整理することを通じて、
フーリエ変換への理解を深めることができる。実際に、授業の冒頭では先にフーリエ変換に関する理論的な理解の確認を行い、その理論的な結果と実験を通して得られた結果を比較した。

\section{理論的背景}

\subsection{連続時間信号と周波数スペクトル}
信号とは、情報を表す量が時間とともに変化するものであり、本実験では連続時間信号に注目する。連続時間信号は単一の周波数だけでなく、さまざまな周波数成分を含んでいる。周期性を持たない信号の周波数成分を明らかにするため、フーリエ変換 $X(\omega)=\int_{-\infty}^{\infty} x(t)\exp(-i\omega t)dt$ を用いて周波数領域に変換する。周波数スペクトル $X(\omega)$ は複素数となるため、グラフ表示には振幅スペクトル $|X(\omega)|$ やエネルギースペクトル $|X(\omega)|^2$ を用いることが多い。

\subsection{アナログ変調の基礎}
下限周波数が 0 Hz である帯域をベースバンドと呼び、0 Hz を含む信号をベースバンド信号 $m(t)$ と呼ぶ。ベースバンド伝送には、一本の伝送路で複数の信号を送れないという制約がある。この制約を解決するため、ベースバンド信号を高い周波数帯へ移す周波数変換(変調)搬送波 $\cos\omega_c t$ を用いて行う。

\subsection{振幅変調 (AM)}
搬送波の振幅をベースバンド信号 $m(t)$ に合わせて変化させる方式であり、変調信号 $x_{AM}(t)$ は $x_{AM}(t)=m(t)\cos\omega_c t$ で表される。AM信号の周波数スペクトル $X_{AM}(\omega)$ は、ベースバンド信号のスペクトル $M(\omega)$ を搬送波周波数 $\omega_c$ の正負にそれぞれ移動させた形となる。このとき、スペクトルの形状は変化しない(線形変調)。

\subsection{位相変調 (PM) とベッセル関数}
搬送波の位相成分をベースバンド信号 $m(t)$ に合わせて直接的に変化させる方式を位相変調 (PM) と呼び、信号 $x_{PM}(t)$ は $x_{PM}(t)=\cos\{\omega_c t+k_p m(t)\}$ で与えられる。 ベースバンド信号 $m(t)=\sin\omega_m t$(正弦波)の場合、PM信号のスペクトル解析にはベッセル関数展開を用いる。その周波数スペクトル $X_{PM}(\omega)$ は、$\omega_c$ を中心に $\omega_m$ の整数倍の間隔 $(\omega_c \pm n\omega_m)$ で線スペクトル(くし形スペクトル)として現れる。このスペクトルの振幅は第 1 種ベッセル関数 $J_n(k_p)$ によって決定され、変調度 $k_p$ が大きいほど高調波成分(n>1)が大きくなる。


\section{実験方法(セットアップと手順)}


\section{実験結果および考察(検討項目の実施)}

\subsection{振幅変調の結果と考察}

\subsubsection{正弦波変調}


\subsubsection{矩形波変調}


\subsubsection{Sinc波変調}


\subsection{位相変調の結果と考察}

\subsubsection{正弦波変調}


\subsubsection{Sinc波変調}


\section{結論/まとめ}


\section{班内での役割と自己貢献}


\section{参考文献}


\end{document}