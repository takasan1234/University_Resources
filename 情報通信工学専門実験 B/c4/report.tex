\documentclass{jlreq}
% \usepackage{xeCJK}
\usepackage[top=2.5cm, bottom=2.5cm, left=2.5cm, right=2.5cm]{geometry}

\pagestyle{empty}


\begin{document}

\begin{center}
    {\huge 電子情報工学科} \\
    \vspace{0.7cm}
    {\huge     工学専門実験  報告} \\ %科目名を記入
    \vspace{1.4cm}
    {\Large 第 1 号} \\ %実験番号を記入
    \vspace{2.1cm}
    {\large 実  験  題  目} \\
    \vspace{1.5cm}
    \underline{\huge C4 信号変調の基礎実験} \\%実験題目を記入
    \vspace{0.7cm}
    \underline{情報システム工学コース   第 2 班} \\%コース名と班を記入
    \underline{\Large阿部 寛希}\\%同じ班の氏名を記入
    \underline{\Large尾本 涼太}\\%同じ班の氏名を記入
    \underline{\Large菊池 圭祐}\\%同じ班の氏名を記入
\end{center}

\vfill

\begin{center}
    {\Large 報 告 者} \\
    \vspace{1cm}
    \underline{\large 08D23091 番 辻孝弥 (情報システム工学コース)} \\%報告者の学籍番号、氏名、コース名を記入
    \vspace{1cm}
    \underline{\large 電子メールアドレス:\hspace{1cm}u316705e@ecs.osaka-u.ac.jp }\\%報告者のメールアドレスを記入
\end{center}

\vfill

\begin{center}
    \large 令和 7 年 11 月 19 日 \\%報告日を記入
    \vspace{0.7cm}
    \Large 大阪大学工学部電子情報工学科
\end{center}

\newpage

\section{目的}


\section{理論的背景(変復調技術の基礎)}


\section{実験方法(セットアップと手順)}


\section{実験結果および考察(検討項目の実施)}

\subsection{振幅変調の結果と考察}

\subsubsection{正弦波変調}


\subsubsection{矩形波変調}


\subsubsection{Sinc波変調}


\subsection{位相変調の結果と考察}

\subsubsection{正弦波変調}


\subsubsection{Sinc波変調}


\section{結論/まとめ}


\end{document}