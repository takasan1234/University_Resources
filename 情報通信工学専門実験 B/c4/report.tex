\documentclass{jlreq}
% \usepackage{xeCJK}
\usepackage[top=2.5cm, bottom=2.5cm, left=2.5cm, right=2.5cm]{geometry}

\pagestyle{empty}


\begin{document}

\begin{center}
    {\huge 電子情報工学科} \\
    \vspace{0.7cm}
    {\huge     工学専門実験  報告} \\ %科目名を記入
    \vspace{1.4cm}
    {\Large 第 1 号} \\ %実験番号を記入
    \vspace{2.1cm}
    {\large 実  験  題  目} \\
    \vspace{1.5cm}
    \underline{\huge C4 信号変調の基礎実験} \\%実験題目を記入
    \vspace{0.7cm}
    \underline{情報システム工学コース   第 2 班} \\%コース名と班を記入
    \underline{\Large 阿部 寛希}\\%同じ班の氏名を記入
    \underline{\Large 尾本 涼太}\\%同じ班の氏名を記入
    \underline{\Large 菊池 圭祐}\\%同じ班の氏名を記入
\end{center}

\vfill

\begin{center}
    {\Large 報 告 者} \\
    \vspace{1cm}
    \underline{\large 08D23091 番 辻孝弥 (情報システム工学コース)} \\%報告者の学籍番号、氏名、コース名を記入
    \vspace{1cm}
    \underline{\large 電子メールアドレス:\hspace{1cm}u316705e@ecs.osaka-u.ac.jp }\\%報告者のメールアドレスを記入
\end{center}

\vfill

\begin{center}
    \large 令和 7 年 11 月 19 日 \\%報告日を記入
    \vspace{0.7cm}
    \Large 大阪大学工学部電子情報工学科
\end{center}

\newpage
\pagestyle{plain}
\pagenumbering{arabic}

\section{目的}
本実験の目的は、ベースバンド信号を振幅変調および位相変調した際の、
時間波形と周波数スペクトルの特徴を理解することである。
これにより、信号を時間領域と周波数領域の対応関係で整理することを通じて、
フーリエ変換への理解を深めることができる。実際に、授業の冒頭では先にフーリエ変換に関する理論的な理解の確認を行い、その理論的な結果と実験を通して得られた結果を比較した。

\section{理論的背景}

\subsection{連続時間信号と周波数スペクトル}
信号とは、情報を表す量が時間とともに変化するものであり、本実験では連続時間信号に注目する。連続時間信号は単一の周波数だけでなく、さまざまな周波数成分を含んでいる。周期性を持たない信号の周波数成分を明らかにするため、フーリエ変換 $X(\omega)=\int_{-\infty}^{\infty} x(t)\exp(-i\omega t)dt$ を用いて周波数領域に変換する。周波数スペクトル $X(\omega)$ は複素数となるため、グラフ表示には振幅スペクトル $|X(\omega)|$ やエネルギースペクトル $|X(\omega)|^2$ を用いることが多い。

\subsection{アナログ変調の基礎}
下限周波数が 0 Hz である帯域をベースバンドと呼び、0 Hz を含む信号をベースバンド信号 $m(t)$ と呼ぶ。ベースバンド伝送には、一本の伝送路で複数の信号を送れないという制約がある。この制約を解決するため、ベースバンド信号を高い周波数帯へ移す周波数変換(変調)搬送波 $\cos\omega_c t$ を用いて行う。

\subsection{振幅変調 (AM)}
搬送波の振幅をベースバンド信号 $m(t)$ に合わせて変化させる方式であり、変調信号 $x_{AM}(t)$ は $x_{AM}(t)=m(t)\cos\omega_c t$ で表される。AM信号の周波数スペクトル $X_{AM}(\omega)$ は、ベースバンド信号のスペクトル $M(\omega)$ を搬送波周波数 $\omega_c$ の正負にそれぞれ移動させた形となる。このとき、スペクトルの形状は変化しない(線形変調)。

\subsection{位相変調 (PM) とベッセル関数}
搬送波の位相成分をベースバンド信号 $m(t)$ に合わせて直接的に変化させる方式を位相変調 (PM) と呼び、信号 $x_{PM}(t)$ は $x_{PM}(t)=\cos\{\omega_c t+k_p m(t)\}$ で与えられる。 ベースバンド信号 $m(t)=\sin\omega_m t$(正弦波)の場合、PM信号のスペクトル解析にはベッセル関数展開を用いる。その周波数スペクトル $X_{PM}(\omega)$ は、$\omega_c$ を中心に $\omega_m$ の整数倍の間隔 $(\omega_c \pm n\omega_m)$ で線スペクトル(くし形スペクトル)として現れる。このスペクトルの振幅は第 1 種ベッセル関数 $J_n(k_p)$ によって決定され、変調度 $k_p$ が大きいほど高調波成分(n>1)が大きくなる。


\section{実験方法(セットアップと手順)}

本実験では、信号の生成、変調、および観測のために、以下の機器を使用した。

\subsection{実験セットアップおよび使用機器}
実験系は、任意波形発生器をオシロスコープ、またはスペクトラムアナライザに接続する構成で構築された。

使用機器は以下の通りである。

\begin{table}[h]
\centering
\begin{tabular}{|l|l|l|l|}
\hline
機器名称 & 型番 & 役割 & 観測対象 \\
\hline
パーソナルコンピュータ & PC & Excelにて波形データを作成 & — \\
\hline
任意波形発生器 & Tektronix AFG1062 & 変調信号の出力 \\
\hline
デジタルオシロスコープ & Tektronix TBS1102B-EDU & 時間領域の波形観測 & 時間波形 \\
\hline
RF スペクトラムアナライザ & Agilent Technology N9060A & 周波数領域の波形観測 & 周波数スペクトル \\
\hline
\end{tabular}
\end{table}

\subsection{共通の搬送波設定}
変調に使用する搬送波には、以下のパラメータを持つ正弦波を使用した。
\begin{itemize}
\item 周期 $T_c$ : $T_c=1\mu$s (周波数 1 MHz に相当)
\item ピーク電圧 $A_c$ : $A_c=1.0$V
\end{itemize}

\subsection{実験手順}
\subsubsection{波形データをPCで作成}
それぞれの実験で必要な変調信号の波形データをMicrosoft Excelを使用して作成した。作成したデータを任意波形発生器(Tektronix AFG1062)に送信し、信号を生成した。その信号の時間波形と周波数スペクトルを、それぞれデジタルオシロスコープ(Tektronix TBS1102B-EDU)、スペクトラムアナライザ(Agilent Technology N9060A)を使用して観測した。

\subsubsection{一週目:振幅変調(AM)}
周期 $T_c=1\mu$s、ピーク電圧 $A_c=1.0$V の正弦波を搬送波とし、以下のベースバンド信号で搬送波の振幅を変調した。

\paragraph{正弦波変調 (a)}
\begin{itemize}
\item 手順 (i): 周期 $T_m=10\mu$s、ピーク電圧 $A_m=1$V の正弦波で搬送波の振幅を変調し、時間波形と周波数スペクトルを観測・保存した。時間波形は、搬送波の高周波成分がみえる時間幅と、変調信号の周期がわかる時間幅の 2 種類を観測・保存した。周波数スペクトル測定では、縦軸のスケールを Linear に設定し、観測・保存した。
\item 手順 (ii): ピーク電圧 $A_m$ を 0.5 V 〜 3 V の範囲で 0.5 V 刻みに変化させ、その際に波形が変化しないことを観測・保存した。そのうえで、周波数スペクトルにおけるピーク値の $A_m$ 依存性を評価するためにデータを取得した。
\end{itemize}

\paragraph{矩形波変調 (b)}
矩形幅 $T_m=10\mu$s、繰り返し周期 $T_r=1$ms、ピーク電圧 $A_m=1$V の矩形波を用いて搬送波の振幅を変調し、時間波形と周波数スペクトルを観測・保存した。

\paragraph{Sinc波変調 (c)}
減衰時間 $T_m=10\mu$s、繰り返し周期 $T_r=1$ms、ピーク電圧 $A_m=1$V の Sinc波形で搬送波の振幅を変調し、時間波形と周波数スペクトルを観測・保存した。

\subsubsection{二週目:位相変調(PM)}
一週目と同様の搬送波を用い、位相変調された信号を観測した。観測時、位相変調信号の時間波形では包絡線は一定であり、その周期が変調されている点に注意を払った。

\paragraph{正弦波変調 (a)}
\begin{itemize}
\item 周期 $T_m=10\mu$s の正弦波で搬送波の位相を変調し、変調信号のピーク電圧 $A_m$ を 3 V 〜 0.5 V の範囲で 0.5 V 刻みに変化させたときの時間波形と周波数スペクトルを観測・保存した。
\item 観測された、1 MHz を中心に 100 kHz 間隔で線スペクトルを持つくし形の周波数スペクトルについて、1 MHz の搬送波成分、1.1 MHz の基本波成分、1.2 MHz の 2 次高調波成分のピーク値を測定し、その $A_m$ 依存性を評価するためのデータを取得した。
\end{itemize}

\paragraph{Sinc波変調 (b)}
減衰時間 $T_m=10\mu$s、繰り返し周期 $T_r=1$ms の Sinc 波を用いて搬送波の位相を変調した。Sinc波形のピーク電圧 $A_m$ を 6 V 〜 1 V の範囲で 1 V 刻みに変化させたときの周波数スペクトルを観測した。なお、時間波形の観測も行ったが、保存は省略した。


\section{実験結果および考察(検討項目の実施)}

\subsection{振幅変調の結果と考察}

\subsubsection{正弦波変調}


\subsubsection{矩形波変調}


\subsubsection{Sinc波変調}


\subsection{位相変調の結果と考察}

\subsubsection{正弦波変調}


\subsubsection{Sinc波変調}


\section{結論/まとめ}


\section{班内での役割と自己貢献}


\section{参考文献}


\end{document}