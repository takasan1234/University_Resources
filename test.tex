\documentclass[a4paper,11pt]{article}

% 日本語設定
\usepackage{plautopatch}
\usepackage[utf8]{inputenc}
\usepackage{CJKutf8}
\usepackage{otf}

% 必要なパッケージ
\usepackage{amsmath,amssymb}
\usepackage{graphicx}
\usepackage{booktabs}
\usepackage{enumitem}
\usepackage{url}

% 丸数字のための定義
\newcommand{\circled}[1]{\textcircled{\scriptsize #1}}

% タイトル設定
\title{COMPASデータセットを用いた再犯予測モデルの精度向上}
\author{工学部 電子情報学科\\学籍番号:08D23091\\辻 孝弥}
\date{\today}

\begin{document}

\maketitle

\section{はじめに}
本課題では,COMPASデータセットを用いて再犯予測モデルの精度向上を目的とした。XGBoostとMLPの2種類のモデルを構築・改善し,さらに両者をスタッキングする手法を適用した。

\section{実験方法}
\subsection{XGBoostモデルの導入とハイパーパラメータ調整}
使用モデル: XGBClassifier

\subsubsection{Optunaによる事前探索と二段階学習}
Optuna を用いて最適なハイパーパラメータを事前に調査し,その結果を \texttt{best\_params} に設定して一次学習を実施した。\par
続いて一次学習済みモデルで派生特徴量を生成し,クラス不均衡補正などを加えた新たなパラメータセット \texttt{new\_params} で再度全データを用いた二次学習を行った。

\subsubsection{主なハイパーパラメータ設定}
\begin{itemize}[nosep]
  \item 学習率 (\texttt{learning\_rate}):0.05 $\rightarrow$ 0.08
  \item 木の深さ (\texttt{max\_depth}):3
  \item サブサンプリング率 (\texttt{subsample}):0.90
  \item 特徴量サンプリング率 (\texttt{colsample\_bytree}):0.78
  \item 最小子ノード重量 (\texttt{min\_child\_weight}):5
  \item 正則化パラメータ (\texttt{gamma}):0.50
  \item L1正則化 (\texttt{reg\_alpha}):$5.4 \times 10^{-6}$
  \item L2正則化 (\texttt{reg\_lambda}):5.88
  \item 木数 (\texttt{n\_estimators}):200
  \item 早期停止 (\texttt{early\_stopping\_rounds}):10
  \item クラス不均衡補正 (\texttt{scale\_pos\_weight}):訓練データのクラス比に基づき自動設定
\end{itemize}

\subsection{特徴量エンジニアリング}
\subsubsection{不要特徴量の削減}
重要度上位80\%を残し,下位20\%を削除(\texttt{priors\_count}・\texttt{age} は保護)

\subsubsection{ChatGPTによる特徴量選択}
全てのCSVカラムを逐次試すのは非効率なため,ChatGPT を用いて平均的に効果が見込める無難な派生特徴量案を取得し,それを見ながら実装した。

\subsubsection{主な派生特徴量}
\begin{itemize}[nosep]
  \item \texttt{priors\_per\_year}(前科数/(年齢+1))
  \item \texttt{sum\_priors\_and\_age}(前科数+年齢)
  \item \texttt{age\_squared}(年齢$^2$)
  \item \texttt{log\_priors\_p1}($\log$(前科数+1))
  \item \texttt{age\_times\_priors}(年齢×前科数)
  \item \texttt{total\_juv\_cnt}(少年期犯罪合計)
  \item \texttt{juv\_ratio}(少年期犯罪合計/(前科数+1))
  \item \texttt{log\_len\_stay}(拘束期間の対数化)
\end{itemize}

\subsubsection{\texttt{dob}(生年月日)の扱い}
\begin{itemize}[nosep]
  \item 初期には再犯率と無関係と判断し除外したが,除外時のAccuracyが0.688$\rightarrow$0.679に低下
  \item 最終的には \texttt{dob} を含める実装とし,Accuracyを0.699$\rightarrow$0.710へ改善
\end{itemize}

\subsection{MLPモデルの構造・学習戦略改善}
\subsubsection{ネットワーク構造}
隠れ層:256→128,ReLU+BatchNorm+Dropout(0.3/0.2)

\subsubsection{学習戦略}
\begin{itemize}[nosep]
  \item Optimizer:Adam(lr=1e-3, weight\_decay=1e-5)
  \item Scheduler:CosineAnnealingLR
  \item 損失関数:クラス重み付きCrossEntropy
  \item EarlyStopping:patience=15
\end{itemize}

\subsection{モデルスタッキング}
\begin{itemize}[nosep]
  \item メタ学習器:LogisticRegression (L2, C=1.0)
  \item 入力特徴:XGBoost/MLP の検証データ予測確率
  \item 閾値決定:Youden's J による最適閾値
\end{itemize}

\section{結果}
\begin{table}[htbp]
  \centering
  \begin{tabular}{lc}
    \toprule
    モデル & Accuracy \\
    \midrule
    \circled{1} MLP 単体 & 0.675 \\
    \circled{2} XGBoost 単体+特徴量変更 & 0.688 \\
    \circled{3} \circled{2}モデル(dob除外) & 0.679 \\
    \circled{4} 今回実装モデル(dob除外) & 0.699 \\
    \circled{5} 完全実装モデル(dob含む) & 0.710 \\
    \bottomrule
  \end{tabular}
\end{table}

\subsection{\circled{5}の詳細ログ}
\subsubsection{混同行列}
\begin{verbatim}
[[1591  421]
 [ 627  968]]
\end{verbatim}

\subsubsection{分類レポート}
\begin{verbatim}
              precision    recall  f1-score   support
           0      0.717     0.791     0.752      2012
           1      0.697     0.607     0.649      1595

    accuracy                          0.710      3607
   macro avg      0.707     0.699     0.701      3607
weighted avg      0.708     0.710     0.706      3607
\end{verbatim}

Accuracy: 0.710\\
ROC-AUC : 0.767\\
LogLoss : 0.576

\section{考察}
\subsection{Optunaによるハイパラ最適化}
一次学習で得られたパラメータをもとに,二次学習時にはクラス不均衡補正などを加えた \texttt{new\_params} を適用し,性能向上に寄与した。

\subsection{ChatGPT活用の特徴量設計}
無難で効果の期待できる特徴量案を迅速に取得でき,実装工数を大幅に削減できた。

\subsection{\texttt{dob}除外の効果検証}
当初「再犯率に関係がない」と判断して \texttt{dob} を除外したが,Accuracyが0.688$\rightarrow$0.679に低下した。\par
\texttt{dob} を含めることで0.699$\rightarrow$0.710へ改善し,生年月日情報が有用であることを確認した。

\subsection{スタッキング効果}
異なるモデルの補完性により,最終的にAccuracy:0.710/ROC–AUC:0.767/LogLoss:0.576を達成した。

\section{結論}
Optunaによる二段階ハイパラ最適化とChatGPT提案の特徴量エンジニアリングを組み合わせ,実装に忠実に \texttt{dob} 情報の有効性を再検証した結果,再犯予測タスクにおいて高い汎化性能を実現できた。今後はさらに異なる情報源やモデル統合手法を探索し,性能向上を図る余地がある。

\section*{参考文献}
\begin{itemize}
  \item スタッキングの実装と効果について \\
  \url{https://potesara-tips.com/ensemble-stacking/#toc13}
\end{itemize}

\end{document}
