\documentclass[a4paper,12pt]{jsarticle}
\usepackage[utf8]{inputenc}
\usepackage[japanese]{babel}
\usepackage{geometry}
\geometry{margin=20mm}
\usepackage{graphicx}
\usepackage{array}
\usepackage{booktabs}

\title{学習者視点の学びを共有するプラットフォーム「ノートモ」}
\author{代表者:辻 孝弥}
\date{}

\begin{document}

\maketitle

% ロゴ挿入指示
\begin{center}
\textbf{※ここに「ノートモのロゴ+キャッチコピー画像」を挿入する}
\end{center}

\section{背景・課題}

\subsection{学習効率の格差}
まとめノートの有無によって学習効率が大きく変わる現実があります。友人から授業のまとめノートを借りることで飛躍的に学習効率が向上する一方、そのようなノートにアクセスできない学習者は遠回りを強いられています。

\subsection{既存情報ソースの限界}
教科書・参考書・授業は専門家視点で作られており、初学者には理解しづらい部分があります。学習者がつまずきやすいポイントや効率的な理解方法が十分に提供されていません。

\subsection{市場の成長性}
\textbf{※ここに【デジタルノート市場規模推移グラフ】を挿入する}

世界のデジタルノート市場は2024年に約1,300億円、2033年には約2,200億円規模へ成長する見込みです。日本でも学習スタイルのデジタル化が急速に進んでおり、従来の紙のノートからデジタル手書きノートへの移行が進んでいます。

\subsection{ユーザーの声}
β版では32名のユーザーが利用し、学習者目線のノート共有による効率化の価値を実感しています。

\newpage

\section{提案内容・差別化ポイント}

\subsection{ノートモの主要機能}
\begin{itemize}
\item \textbf{学習者作成ノートのアップロード/閲覧}:リアルな学習者視点のノートを共有
\item \textbf{ポイント付与&報酬還元機能}:投稿者へのインセンティブ設計
\item \textbf{モバイル/Web対応}:いつでもどこでもアクセス可能
\item \textbf{コメント・いいね・タグによるコミュニティ形成}:学習者同士の交流促進
\end{itemize}

\subsection{差別化ポイント}
\begin{itemize}
\item \textbf{学習者目線での「つまずき→克服のプロセス」可視化}:初学者が理解しやすい情報提供
\item \textbf{投稿者体験の価値化}:これまで「自分だけの経験」だった学びの知見を価値として認める
\item \textbf{日本初の本格プラットフォーム}:海外のCourse Hero市場(年間約1,000億円規模)に対し、国内には同様の大規模プラットフォームが存在しない
\end{itemize}

\subsection{機能比較}
\begin{table}[h]
\centering
\begin{tabular}{|l|c|c|c|}
\hline
機能 & 既存サービス A & 既存サービス B & ノートモ \\
\hline
学習者目線ノート & × & △ & ○ \\
\hline
報酬ポイント機能 & × & × & ○ \\
\hline
コミュニティ機能 & △ & ○ & ○ \\
\hline
\end{tabular}
\end{table}

\textbf{※ここに【UXフロー図(つまずき発見→ノート閲覧→つまずき克服→ポイント獲得)】を挿入する}

\section{効果・将来性}

\subsection{初年度目標}
\begin{itemize}
\item \textbf{会員登録4万人}
\item \textbf{売上35万円}(0.3%有料化率、平均継続0.75ヶ月)
\end{itemize}

\subsection{5年後ビジョン}
\begin{itemize}
\item \textbf{会員数50万人中3%(約1.5万人)を有料化}
\item \textbf{サブスク収益約4,050万円}(月額900円×継続3ヶ月)
\item \textbf{ポイント課金約276万円}(ノート1冊120円)
\item \textbf{広告収入約1,800万円}
\item \textbf{合計約6,126万円の売上見込み}
\end{itemize}

\textbf{※ここに【収益推移積み上げ棒グラフ】を挿入する}

\textbf{※ここに【KPI達成マイルストーン図】を挿入する}

\subsection{社会インパクト}
学習の無駄・孤独を減らし、日本の教育効率化に寄与します。学習者同士が互いに支え合うコミュニティ型プラットフォームにより、効率的で楽しい学びを実現します。

\subsection{審査基準との関連}
\begin{itemize}
\item \textbf{情熱}:自身の体験から生まれた強い想い
\item \textbf{成長性}:拡大するデジタルノート市場での先行者利益
\item \textbf{経営者適性}:6名チームでの開発実績とβ版運営
\item \textbf{その他特徴}:日本初の学習者目線特化プラットフォーム
\end{itemize}

\end{document}
