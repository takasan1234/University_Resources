\documentclass[a4paper,11pt]{article}

% 日本語設定
\usepackage{plautopatch}
\usepackage[utf8]{inputenc}
\usepackage{CJKutf8}
\usepackage{otf}

% 必要なパッケージ
\usepackage{amsmath,amssymb}
\usepackage{graphicx}
\usepackage{enumitem}
\usepackage{url}
\usepackage{listings}
\usepackage{xcolor}

% コードスタイル設定
\lstset{
  basicstyle=\ttfamily\small,
  frame=single,
  breaklines=true,
  commentstyle=\color{green!50!black},
  keywordstyle=\color{blue},
  stringstyle=\color{red},
  showstringspaces=false
}

% タイトル設定
\title{機械学習に関する基礎知識}
\author{工学部 電子情報学科\\学籍番号:08D23091\\辻 孝弥}
\date{\today}

\begin{document}

\maketitle


\section{Accuracy, Precision, Recall, F1-score の計算式}

混同行列(Confusion Matrix)の要素:
\begin{itemize}
    \item TP (True Positive): 真陽性 - 正しく正と予測された数
    \item TN (True Negative): 真陰性 - 正しく負と予測された数
    \item FP (False Positive): 偽陽性 - 誤って正と予測された数
    \item FN (False Negative): 偽陰性 - 誤って負と予測された数
\end{itemize}

\subsection{Accuracy(正解率)}
\begin{equation}
    \text{Accuracy} = \frac{TP + TN}{TP + TN + FP + FN}
\end{equation}

説明:全体のデータの中で、正しく予測できた割合を表す。

\subsection{Precision(適合率)}
\begin{equation}
    \text{Precision} = \frac{TP}{TP + FP}
\end{equation}

説明:正と予測したデータの中で、実際に正であった割合を表す。

\subsection{Recall(再現率)}
\begin{equation}
    \text{Recall} = \frac{TP}{TP + FN}
\end{equation}

説明:実際に正であるデータの中で、正と予測できた割合を表す。

\subsection{F1-score(F1値)}
\begin{equation}
    \text{F1-score} = 2 \cdot \frac{\text{Precision} \cdot \text{Recall}}{\text{Precision} + \text{Recall}}
\end{equation}

説明:Precision と Recall の調和平均である。両者のバランスを考慮した指標で、どちらか一方だけが極端に高い場合に低い値になる。クラスの偏りが大きいデータセットでAccuracyよりも有用である。

\section{deepcopy と copy 関数の違い}

\begin{itemize}
    \item \textbf{copy}: オブジェクトの参照をコピーする。元のオブジェクトとコピーされたオブジェクトは、同じデータを参照するため、一方を変更するともう一方も変更される。
    
    \item \textbf{deepcopy}: オブジェクトとその内部のオブジェクトも含めて、すべてを新たにコピーする。元のオブジェクトとコピーされたオブジェクトは完全に独立しており、一方を変更してももう一方に影響はない。
\end{itemize}

\section{optimizer.zero\_grad() の有無による動作の違い}

\texttt{optimizer.zero\_grad()} は、モデルのパラメータの勾配を0にリセットする関数である。

\subsection{optimizer.zero\_grad() がある場合}
各 iteration(バッチ)の前に勾配が0にリセットされるため、勾配が累積されずに、現在のバッチのデータに基づいてパラメータが更新される。

\subsection{optimizer.zero\_grad() がない場合}
\begin{itemize}
    \item 勾配が iteration 毎に累積されていく。
    \item 過去の勾配情報が現在の更新に影響を与えるため、学習が不安定になったり、意図した方向に学習が進まなかったりする可能性がある。
\end{itemize}

\subsection{影響}
\texttt{optimizer.zero\_grad()} を実行しないと、以下のような問題が生じる可能性がある:
\begin{itemize}
    \item loss が発散しやすくなる
    \item モデルの学習速度が遅くなる
    \item モデルが収束しなくなる
\end{itemize}

\subsection{結論}
\texttt{optimizer.zero\_grad()} は、各 iteration の前に勾配をリセットすることで、安定した学習を保証するために必要な処理である。

\section*{参考文献}
\begin{itemize}
  \item Google Developers: Machine Learning Crash Course - Classification: Accuracy, Precision, Recall \\
  \url{https://developers.google.com/machine-learning/crash-course/classification/accuracy-precision-recall?hl=ja}
\end{itemize}

\end{document}
